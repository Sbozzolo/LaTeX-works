\documentclass[]{scrartcl}
\usepackage[italian]{babel}
\usepackage[utf8]{inputenc}
\usepackage{amsmath, amsfonts}
\usepackage{cases}
\usepackage{braket}
\usepackage{tikz, pgfplots}
\usepackage[makeroom]{cancel}


\usepgfplotslibrary{fillbetween}
\usetikzlibrary{patterns}
\pgfdeclarepatternformonly{north east lines wide}%
{\pgfqpoint{-1pt}{-1pt}}%
{\pgfqpoint{10pt}{10pt}}%
{\pgfqpoint{9pt}{9pt}}%
{
	\pgfsetlinewidth{0.4pt}
	\pgfpathmoveto{\pgfqpoint{0pt}{0pt}}
	\pgfpathlineto{\pgfqpoint{9.1pt}{9.1pt}}
	\pgfusepath{stroke}
}

\newcommand{\qedsymbol}{\hfill \rule{0.7em}{0.7em}}

\usepackage[]{xcolor}  
\usepackage[colorinlistoftodos,prependcaption,textsize=tiny]{todonotes}

\newcommand{\px}{\partial_x}
\newcommand{\py}{\partial_y}
\newcommand{\pt}{\partial_t}
\newcommand{\pxy}{\partial_x\partial_y}
\newcommand{\pdx}[1]{\partial^#1_x}
\newcommand{\pdy}[1]{\partial^#1_y}
\newcommand{\pdt}[1]{\partial^#1_t}
\newcommand{\xvt}{(\mathbf{x},t)}
\newcommand{\xt}{(x,t)}
\newcommand{\lap}{\nabla^2}

% Title Page
\title{Esercizi di Equazioni Differenziali}
\author{Gabriele Bozzola \\ Matricola: 882709}
\date{8 Marzo 2016}

\begin{document}
\maketitle


\section*{Esercizio 1}

\subsection*{Testo} Equazione del calore

\begin{enumerate}
	\item Risolvere l'equazione del calore con sorgente:
	\begin{equation}
	\pt T\xvt = \kappa \lap T \xvt + S \xvt, \qquad t \geq 0,
	\label{eq:calore}
	\end{equation}
	nel quadrato $ 0 \leq x \leq 1 ,\, 0 \leq y \leq 1 $ con le condizioni iniziali $ T(\mathbf{x},0) = T_0(\mathbf{x}) $ e le condizioni al contorno $ T\xvt = 0 $ sul bordo $ x=0,~x=1,~y=0,~y=1 $.
	\item Stessa cosa, ma senza sorgente ($ S\xvt = 0 $) e con pareti del quadrato isolato, cioè le condizioni al contorno diventano di Neumann.
	\[	\px T\xvt = 0 \quad \text{per} \quad x = 0,\,x=1, \qquad \py T\xvt = 0 \quad \text{per} \quad y = 0,\,y=1,	\]
	(Nessun flusso termico attraverso le pareti).
\end{enumerate}

\subsection*{Svolgimento} 
\subsubsection*{Punto 1}
Questo problema può essere risolto con l'utilizzo del kernel di calore in modo che la soluzione di \eqref{eq:calore} sia:  
\begin{multline}
T(x,y;t) = \int_0^t \int_0^1 \int_0^1 G(t-t';x,y;z,w) S(z,w;t')\,dz\,dw\,dt' + \\ \int_0^1 \int_0^1 T_0(z,w) G(t;x,y;z,w)\,dz\,dw
\label{eq:sol_general_calore}
\end{multline}
Con 
\begin{equation}
G(t-t';x,y;z;w) = \sum_{nm} \mathrm{e}^{-\kappa \lambda_{nm} (t-t')} \varphi_{nm}(x,y) \varphi_{nm}(z,w)
\label{eq:green_calore}
\end{equation}
Dove le $ \{\varphi_{nm}\} $ sono le autofunzioni del Laplaciano nel quadrato, ovvero le soluzioni di: 
  \begin{equation}
  	\begin{cases}
  	\lap \varphi_{nm}(x,y) + \lambda_{nm}\varphi_{nm}(x,y) = 0 & \text{per} \quad 0\leq x \leq 1,\; 0\leq y \leq 1  \\
  	\varphi_{nm}(x,y) = 0 & \text{per} \quad x = 0 \vee x = 1 \vee y = 0 \vee y = 1 \label{eq:autofunzioni_laplaciano}
  	\end{cases}{}
  \end{equation}
Il problema è quello di una particella libera in un quadrato, risolto da funzioni del tipo:
\[	\varphi_{nm}(x,y) \propto \sin\left(k_x x\right)\sin\left(k_y y\right)	\]
Tali $ \varphi_{nm}(x,y) $ risolvono \eqref{eq:autofunzioni_laplaciano} se $ k_x = n\pi $ e $ k_y = m\pi $ con $ n,m \in \{1,\,2,\,\dots\} $. La costante di proporzionalità è fissata a $ 2 $ dalla condizione di normalizzazione. Gli autovalori sono quindi $ \pi^2(n^2+m^2) $, perciò la funzione di Green è per \eqref{eq:green_calore}:
\[	G(t-t';x,y;z;w) = 4 \sum_{nm} \mathrm{e}^{-\kappa \pi^2(n^2 + m^2) (t-t')} \sin(n\pi x) \sin(m \pi y) \sin(n\pi z) \sin(m \pi w)	\]
Utilizzando questa funzione di Green la soluzione del problema è:
\begin{equation}
\begin{split}
& T(x,y;t) = 4 \int_0^1 \int_0^1 \sum_{nm} \mathrm{e}^{-\kappa \pi^2(n^2 + m^2) t} \sin(n\pi x) \sin(m \pi y) \sin(n\pi z) \sin(m \pi w) T_0(z,w)\,dz\,dw +\\
& 4 \int_0^t \int_0^1 \int_0^1 \sum_{nm} \mathrm{e}^{-\kappa \pi^2(n^2 + m^2) (t-t')} \sin(n\pi x) \sin(m \pi y) \sin(n\pi z) \sin(m \pi w) S(z,w;t')\,dt'\,dz\,dw 
\label{eq:seconda_parte_calore}
\end{split}
\end{equation}
%Sviluppando $ T_0(z,w) $ sulla base di Fourier nel quadrato:
%\[	T_0(z,w) = \sum_{nm}^{+\infty} b_{nm} \sin(n \pi z) \sin(m \pi w) \]
%Con:
%\[ b_{nm} = 4 \int_0^1 \int_{0}^{1} T_0(x,y) \sin(n \pi x) \sin(m \pi y)\,dx\,dy	\]
%In questo modo risulta che \eqref{eq:seconda_parte_calore} è uguale a:
%\[ \sum_{nm}^{+\infty} \mathrm{e}^{-\kappa \pi^2(n^2 + m^2) (t-t')} b_{nm} \sin(n\pi x) \sin(m \pi y)	\]
%Tale espressione è la soluzione del problema omogeneo, a cui bisogna aggiungere la soluzione particolare dell'equazione non omogenea:

\subsubsection*{Punto 2}
Si possono utilizzate le tecniche già sfruttate nel punto precedente in modo che la soluzione di \eqref{eq:calore} in assenza di sorgenti con condizioni al contorno di Neumann sia:
\begin{equation}
T(x,y;t) = \int_0^1 \int_0^1 T_0(z,w) G(t;x,y;z,w)\,dz\,dw
\label{eq:sol_general_neumann}
\end{equation}
Con funzione di Green sempre data da \eqref{eq:green_calore}, ma con differenti autofunzioni rispetto a quelle trovate precedentemente. Bisogna quindi risolvere il problema:
  \begin{equation}
  \begin{cases}
  \lap \varphi_{nm}(x,y) + \lambda_{nm}\varphi_{nm}(x,y) = 0 & \text{per} \quad 0\leq x \leq 1,\; 0\leq y \leq 1  \\
  \px\varphi_{nm}(x,y) = 0 & \text{per} \quad x = 0 \vee x = 1 \\
  \py\varphi_{nm}(x,y) = 0 & \text{per} \quad y = 0 \vee y = 1  \label{eq:autofunzioni_laplaciano_neumann}
  \end{cases}{}
  \end{equation}
Un possibile ansatz di soluzione è:
\[	\varphi_{nm}(x,y) \propto \cos\left(k_x x\right)\cos\left(k_y y\right)	\]
Da cui:
\[	\px\varphi_{nm}(x,y) \propto \sin\left(k_x x\right)\cos\left(k_y y\right) \qquad \py\varphi_{nm}(x,y) \propto \cos\left(k_x x\right)\sin\left(k_y y\right)	\]
Tale ansatz risolve \eqref{eq:autofunzioni_laplaciano_neumann} se $ k_x = n\pi $ e $ k_y = m\pi $ con $ n,m \in \{1,\,2,\,\dots\} $. La costante di proporzionalità è fissata a $ 2 $ dalla condizione di normalizzazione. Gli autovalori sono quindi ancora $ \pi^2(n^2+m^2) $. La funzione di Green risulta perciò per \eqref{eq:green_calore}:
\[	G(t-t';x,y;z;w) = 4 \sum_{nm} \mathrm{e}^{-\kappa \pi^2(n^2 + m^2) (t-t')} \cos(n\pi x) \cos(m \pi y) \cos(n\pi z) \cos(m \pi w)	\]
Da cui la soluzione:
\begin{equation}
T(x,y;t) = 4 \int_0^1 \int_0^1 \sum_{nm} \mathrm{e}^{-\kappa \pi^2(n^2 + m^2) t} \cos(n\pi x) \cos(m \pi y) \cos(n\pi z) \cos(m \pi w) T_0(z,w)\,dz\,dw
\label{eq:soluz_calore_neumann}
\end{equation}
\qedsymbol
\section*{Esercizio 2}

\subsection*{Testo} Problema ai valori iniziali e ai valori al contorno

\begin{enumerate}
	\item Risolvere il seguente problema ai valori iniziali:
	\begin{equation}
	u'' + u' -2u = \mathrm{e}^x \qquad x>0, \quad u(0) = 1, \quad u'(0) = 0,
	\label{eq:iniz1}
insta	\end{equation}
	costruendo la funzione di Green one-sided $ R(x,t) $ e integrando successivamente.
	\item Risolvere il seguente problema ai valori iniziali:
	\begin{equation}
	u'' - u = \mathrm{e}^x \qquad 0<x<1, \quad u(0) = 1, \quad u'(1) = 0,
	\label{eq:iniz2}
	\end{equation}
	costruendo la funzione di Green  $ G(x,t) $.
\end{enumerate}

\subsection*{Soluzione}
\subsubsection*{Punto 1}
L'equazione \eqref{eq:iniz1} è un'equazione differenziale ordinaria lineare non omogenea a coefficienti costanti, la cui soluzione è la combinazione lineare di due soluzioni linearmente indipendenti dell'omogenea associata sommata con una soluzione particolare. Il polinomio caratteristico di \eqref{eq:iniz1} è $ \lambda^2 + \lambda - 2 $, che ha radici $ \lambda = -2 $ e $ \lambda = 1 $. L'omogenea è risolta quindi da soluzioni della forma:
\[ A \mathrm{e}^{-2x}	+ B \mathrm{e}^{x} 	\]
Cercando una soluzione particolare del tipo $ Cx\mathrm{e}^x $:
\[2C\mathrm{e}^x + \cancel{Cx\mathrm{e}^x} + C\mathrm{e}^x + \cancel{Cx\mathrm{e}^x} - \cancel{2Cx\mathrm{e}^x} = \mathrm{e}^x \quad \Rightarrow \quad 3C\mathrm{e}^x =\mathrm{e}^x  \quad \Rightarrow \quad C=\frac{1}{3}  \]
Per questo la soluzione generale di \eqref{eq:iniz1} è:
\begin{equation}
u(x) = A \mathrm{e}^{-2x}	+ B \mathrm{e}^{x} + \frac{1}{3}x\mathrm{e}^x
\label{eq:soluz_generale_costanti_1}
\end{equation}
Le costanti $ A $ e $ B $ sono fissate dalle condizioni al contorno: $ A = \frac{4}{9} $ e $ B = \frac{5}{9} $, ottenendo la soluzione:
\begin{equation}
u(x) = \frac{4}{9} \mathrm{e}^{-2x}	+ \frac{5}{9} \mathrm{e}^{x} + \frac{1}{3}x\mathrm{e}^x
\label{eq:soluz_generale_costanti_completa_1}
\end{equation}
L'esercizio richiede di risolvere l'equazione differenziale utilizzando la funzione di Green one-sided $ R(x,t) $. Siano $ a(x) = 1,\, b(x) = 1,\, c(x)=-2,\, F(x) = \mathrm{e}^x $, si definiscono:
\[ p(x) = \exp\left( \int_0^x \frac{b(t)}{a(t)}	\right)	= \mathrm{e}^x \qquad q(x) = \frac{c(x)}{a(x)}p(x) = -2\mathrm{e}^x \qquad f(x) = \frac{F(x)}{a(x)}p(x) = \mathrm{e}^{2x} \]
sicché l'equazione \eqref{eq:iniz1} si può scrivere equivalentemente in termini delle variabili appena definite:
\begin{equation}
\frac{d}{dx}\left( p \frac{du}{dx}	\right) + qu = f \quad \Rightarrow \quad \frac{d}{dx}\left( \mathrm{e}^x \frac{du}{dx}	\right) -2\mathrm{e}^xu = \mathrm{e}^{-2x}
\label{eq:green_one_sided_equazione}
\end{equation} 
La soluzione di \eqref{eq:iniz1}, a meno delle condizioni al contorno, è data da:
\begin{equation}
u(x) = \int_0^x R(x,t) \mathrm{e}^t dt
\label{eq:soluz_generale_1}
\end{equation}
L'equazione omogenea associata a \eqref{eq:green_one_sided_equazione} è, indicando con il punto la derivata rispetto $ x $:
\[ \cancel{\mathrm{e}^x}\dot{v} + \cancel{\mathrm{e}^x}\ddot{v}-2\cancel{\mathrm{e}^x}v = 0 \quad \Rightarrow \quad \ddot{v} + \dot{v} - 2v = 0 \]
Le sue soluzioni $ v_1 $ e $ v_2 $ permettono di costruire la funzione di Green one-sided. Utilizzando la tecnica del polinomio caratteristico si trova che le soluzioni e le sue derivate sono:
\[
\begin{array}{ll}
 v_1 = \mathrm{e}^x & v_1' = \mathrm{e}^x \\
 v_2 = \mathrm{e}^{-2x} & v_2' = -2\mathrm{e}^{-2x}
\end{array}
\]
La funzione di Green one-sided è quindi:
\[	R(x,t) = \frac{v_1(x)v_2(t) - v_2(x)v_1(t))}{p(x)\left(v_1'(x)v_2(x) - v_2'(x)v_1(x)\right)} = \frac{1}{3}\left(\mathrm{e}^x\mathrm{e}^{-2t} - \mathrm{e}^{-2x}\mathrm{e}^t\right)	\]
La quale integrata come in \eqref{eq:soluz_generale_1} restituisce la soluzione dell'equazione \eqref{eq:iniz1} a meno dei termini per correggere le condizioni iniziali. Si ottiene perciò:
\[
\begin{split} 
\tilde{u}(x) & = \frac{1}{3} \int_0^x \left( \mathrm{e}^x \mathrm{e}^{-2t} - \mathrm{e}^{-2x} \mathrm{e}^t \right)\mathrm{e}^{2t}\,dt = \frac{1}{3} \left( \mathrm{e}^{x} \int_0^xdt - \mathrm{e}^{-2x}\int_{0}^{x}\mathrm{e}^{3t}dt\right) = \\ & = \frac{1}{3} \left(x\mathrm{e}^x - \frac{\mathrm{e}^x}{3} + \frac{\mathrm{e}^{-2x}}{3}\right)
\end{split}
\]
Aggiungendo a questa $ u(x) $ un opportuna soluzione dell'omogenea associata a \eqref{eq:iniz1}:
\[ u(x) = \frac{1}{3} \left(x\mathrm{e}^x - \frac{\mathrm{e}^x}{3} + \frac{\mathrm{e}^{-2x}}{3}\right) + Av_1(x) + Bv_2(x) \]
Imponendo le condizioni al contorno si trova $ A=\frac{2}{3} $ e $ B=\frac{1}{3} $ e si riottiene perciò \eqref{eq:soluz_generale_costanti_completa_1}:
\begin{equation}
u(x) = \frac{4}{9} \mathrm{e}^{-2x}	+ \frac{5}{9} \mathrm{e}^{x} + \frac{1}{3}x\mathrm{e}^x
\end{equation}
\subsubsection*{Punto 2}
Come nel caso precedente l'equazione \eqref{eq:iniz2} può essere risolta senza utilizzare la funzione di Green. Il polinomio caratteristico dell'omogenea associata è $ \lambda^2 -1 $ che ha radici $ \lambda = 1 $ e $ \lambda = -1 $. Per questo motivo, se $ A $ e $ B $ sono costanti di integrazione che saranno fissate dalle condizioni al contorno, la soluzione generale dell'omogenea associata è:
\[	A\mathrm{e}^x + B\mathrm{e}^{-x}	\]
Cercando una soluzione particolare del tipo $ Cx\mathrm{e}^x $:
\[2C\mathrm{e}^x + \cancel{Cx\mathrm{e}^x} - \cancel{Cx\mathrm{e}^x} = \mathrm{e}^x \quad \Rightarrow \quad 2C\mathrm{e}^x = \mathrm{e}^x \quad \Rightarrow \quad C=\frac{1}{2}  \]
La soluzione generale di \eqref{eq:iniz2} è perciò 
\begin{equation}
u(x) = A\mathrm{e}^{x} + B \mathrm{e}^{-x} + \frac{1}{2}x\mathrm{e}^x
\label{eq:soluz_generale_costanti_2}
\end{equation}
Le costanti sono fissate dalle condizioni iniziali a $ A = \frac{1-\mathrm{e}^2}{1+\mathrm{e}^2} $ e $ B = \frac{2\mathrm{e}^2}{1+\mathrm{e}^2} $, perciò la soluzione completa di \eqref{eq:iniz2} è 
\begin{equation}
u(x) = \frac{2\mathrm{e}^2}{1+\mathrm{e}^2}  \mathrm{e}^{-x} +  \frac{1-\mathrm{e}^2}{1+\mathrm{e}^2}\mathrm{e}^x + \frac{1}{2}x\mathrm{e}^x
\label{eq:soluz_generale_costanti_completa_2}
\end{equation}
La medesima soluzione può essere ottenuta con il metodo della funzione di Green two-sided.
L'equazione \eqref{eq:iniz2} ha già la forma $ \frac{d}{dx}\left( p \frac{du}{dx}	\right) + qu = -f $ con $ p=1,\,q=-1 $ e $ f(x)=-\mathrm{e}^x $, e la funzione di Green two-sided risolve l'omogenea associata:
\begin{equation}
G''(x,t)-G(x,t)=0
\label{eq:green_two_sieded}
\end{equation}
Le cui soluzioni sono:
\begin{equation}
\begin{cases}
G(x,t) = A(t)\mathrm{e}^x + B(t)\mathrm{e}^{-x} & \text{per} \quad x<t  \\
G(x,t) = C(t)\mathrm{e}^x + D(t)\mathrm{e}^{-x}  & \text{per} \quad x>t  
\label{eq:green_two_sided}
\end{cases}{}
\end{equation}
Imponendo la continuità di $ G(x,t) $ e la discontinuità di $ \frac{dG}{dx}(x,t) $:
\[
\begin{cases}
G(t+\varepsilon,t) = G(t-\varepsilon,t) \Rightarrow A(t)\mathrm{e}^t + B(t)\mathrm{e}^{-t} = C(t)\mathrm{e}^t + D(t)\mathrm{e}^{-t} \\
\frac{dG}{dx}(t+\varepsilon,t) - \frac{dG}{dt}(t-\varepsilon,t) = -1 \Rightarrow A(t)\mathrm{e}^t - B(t)\mathrm{e}^{-t} = -1 + C(t)\mathrm{e}^t - D(t)\mathrm{e}^{-t}
\end{cases}{}
\]
Imponendo le condizioni al contorno:
\[
\begin{cases}
G(0,t) = 0 \Rightarrow A(t)  + B(t) = 0 \Rightarrow B(t) = -A(t) \\
\frac{dG}{dx}(1,t) = 0 \Rightarrow C(t)\mathrm{e}  - \frac{D(t)}{e} = 0 \Rightarrow D(t) = \mathrm{e}^2C(t) \\
\end{cases}{}
\]
Raccogliendo tutte le condizioni trovate:
\[
\begin{cases}
A(t)\mathrm{e}^t + B(t)\mathrm{e}^{-t} = C(t)\mathrm{e}^t + D(t)\mathrm{e}^{-t}\\
A(t)\mathrm{e}^t - B(t)\mathrm{e}^{-t} =  -1 + C(t)\mathrm{e}^t - D(t)\mathrm{e}^{-t}\\
B(t) = -A(t) \\
D(t) = \mathrm{e}^2C(t) \\
\end{cases}{}
\]
Dopo noiosi conti algebrici si trova che:
\[
\begin{cases}
A(t) = \frac{\mathrm{e}}{1+\mathrm{e}^2}\cosh(t-1)\\
B(t) = -\frac{\mathrm{e}}{1+\mathrm{e}^2}\cosh(t-1)\\
C(t) = \frac{1}{1+\mathrm{e}^2}\sinh t\\
D(t) = \frac{\mathrm{e}^2}{1+\mathrm{e}^2}\sinh t \\
\end{cases}{}
\]
In questo modo da \eqref{eq:green_two_sided} si ottiene un'espressione per la funzione di Green two-sided.
\begin{equation}
\begin{cases}
G(x,t) = \frac{\mathrm{e}}{1+\mathrm{e}^2}\cosh(t-1)\left(\mathrm{e}^x - \mathrm{e}^{-x}\right) & \text{per} \quad x<t \\
G(x,t) = \frac{1}{1+\mathrm{e}^2}\sinh t \left( \mathrm{e}^x + \mathrm{e}^2\mathrm{e}^{-x}\right)  & \text{per} \quad x>t 
\label{eq:green_two_sided_trovata}
\end{cases}{}
\end{equation}
La soluzione di \eqref{eq:iniz2} è perciò:
\begin{equation} 
\begin{split}
u(x) & = -\int_0^1 G(x,t) \mathrm{e}^x dt + \frac{\partial G}{\partial t}(x,0) = -\frac{1}{1+\mathrm{e}^2}\int_0^x\sinh t(\mathrm{e}^x + \mathrm{e}^2\mathrm{e}^{-x})\mathrm{e}^t dt - \\ & \frac{\mathrm{e}}{1+\mathrm{e}^2}\int_x^1\cosh(t-1)(\mathrm{e}^x - \mathrm{e}^{-x})\mathrm{e}^t dt +  \frac{1}{1+\mathrm{e^2}} \left( \mathrm{e}^x + \mathrm{e}^2\mathrm{e}^{-x}\right) =  [\dots] =  \\ & = \frac{2\mathrm{e}^2}{1+\mathrm{e}^2}  \mathrm{e}^{-x} +  \frac{1-\mathrm{e}^2}{1+\mathrm{e}^2}\mathrm{e}^x + \frac{1}{2}x\mathrm{e}^x
\end{split}	
\end{equation}
Tale soluzione è esattamente \eqref{eq:soluz_generale_costanti_completa_2}.
\qedsymbol
\section*{Esercizio 3}

\subsection*{Testo} Classificazione di equazioni differenziali quasilineari

\begin{enumerate}
	\item Determinare le regioni del piano $ x-y $ in cui l'equazione differenziale
	\begin{equation}
	2x\pdx{2}u + 2xy\pxy u + y \pdy{2}u + x^4\px u + 2u = \cos x
	\label{eq:2xuxx}
	\end{equation}
	è iperbolica, parabolica o ellittica. Illustrare il risultato con un disegno.
	\item Determinare se l'equazione quasilineare di Burgers inviscida con forza esterna 
	\begin{equation}
	\pt u + u \px u = F\xt
	\label{eq:burgers}
	\end{equation}
	è iperbolica, parabolica o ellittica.
	\item Si consideri l'equazione differenziale:
	\begin{equation}
	\pdx{3}u+y^3\pdy{3}u+\mathrm{e}^{x+2z}\pdy{2}u + \frac{1}{z} \pdx{2} \partial_z u + \sinh(xy^2)\partial_z u= \ln z
	\label{eq:bestia}
	\end{equation}
	\begin{itemize}
		\item Determinare la parte principale.
		\item Determinare la varietà caratteristica $ S(x,y,z) = 0 $.
	\end{itemize}
\end{enumerate}


\subsection*{Svolgimento}
\subsubsection*{Punto 1}
La classificazione delle equazioni differenziali del secondo ordine è basata sul segno del discriminante della parte principale dell'equazione, in questo caso:
	\begin{equation}
	\delta(x,y) = (xy)^2 - 2x y = xy\left(xy-2\right)
	\label{eq:discriminante}
	\end{equation}
L'equazione \eqref{eq:2xuxx} è parabolica dove $ \delta(x,y) = 0 $, cioè nell'insieme $ P $ definito da:
	\begin{equation}
		P = \set{ (x,y) \in \mathbb{R}^2 | x = 0~ \vee y = 0 \vee xy = 2}  
		\label{eq:parabolica}
	\end{equation}
L'equazione \eqref{eq:2xuxx} è iperbolica dove $ \delta(x,y) > 0 $, cioè nell'insieme $ I $ definito da:
	\begin{multline}
	I = \set{ (x,y) \in \mathbb{R}^2 | x > 0 ~ \wedge xy > 2 }  \cup \set{ (x,y) \in \mathbb{R}^2 | x < 0 ~ \wedge xy < 2 } \cup \\
	\set{ (x,y) \in \mathbb{R}^2 | x < 0 ~ \wedge y > 0 }  \cup \set{ (x,y) \in \mathbb{R}^2 | x > 0 ~ \wedge y < 0 }
	\label{eq:iperbolica}
	\end{multline}
L'equazione è ellittica nel restante insieme $ E $, definito da $ E = \mathbb{R}^2 - P - I$.

	\begin{figure}[htbp]
		\centering
		\begin{tikzpicture}
			\begin{axis}[
			xmin=-5, xmax=5,
			ymin=-5, ymax=5,
			axis lines=center,
			axis on top=true,
			xlabel=$ x $,
			ylabel=$ y $,
			ticks=none,
			]
			
				\addplot[mark=none,draw=blue,ultra thick] {0.015};
				\addplot[domain=-5:-0.015, name path=A1] {6};
				\addplot[domain=-5:-0.015, name path=A2] {0};
				\addplot[domain=0.015:5, name path=B1] {-6};
				\addplot[domain=0.015:5, name path=B2] {0};
				\addplot[mark=none,draw=blue,ultra thick, domain=0.1:5, samples=50, name path=C1] {2/x};
				\addplot[mark=none,draw=blue,ultra thick, domain=0.1:5, samples=50, name path=C2] {25/x};
				\addplot[mark=none,draw=blue,ultra thick, domain=-5:-0.1, samples=50, name path=D1] {2/x};
				\addplot[mark=none,draw=blue,ultra thick, domain=-5:-0.1, samples=50, name path=D2] {25/x};
			    \addplot[mark=none, draw=blue, ultra thick] coordinates {(0.015, -5) (0.015, 5)};
				      
		        \addplot[pattern=north east lines wide, pattern color=black!50] fill between[of=A1 and A2];
		        \addplot[pattern=north east lines wide, pattern color=black!50] fill between[of=B1 and B2];
		        \addplot[pattern=north east lines wide, pattern color=black!50] fill between[of=C1 and C2];
		        \addplot[pattern=north east lines wide, pattern color=black!50] fill between[of=D1 and D2];
		        		        	
		        \addplot[cyan!40] fill between[of=A2 and D1];
		        \addplot[cyan!40] fill between[of=B2 and C1];
		        											        
			\end{axis}
		\end{tikzpicture}
		\caption{Classificazione della funzione al variare di $ x $ e $ y $. L'insieme $ P $ è rappresentato in blu (linea continua), l'insieme $ I $ è rappresentato in nero (tratteggiato) e l'insieme $ E $ in azzurro (campitura uniforme)}
	\end{figure}
	
\subsubsection*{Punto 2}
Siccome i coefficienti delle derivate del secondo ordine dell'equazione di Burgers inviscida sono identicamente nulli, il suo discriminante è nullo, quindi è parabolica. 
\subsubsection*{Punto 3}
La parte principale dell'equazione \eqref{eq:bestia} è la somma dei termini contenenti associati le derivate di ordine $ 3 $, cioè è $ \pdx{3}u + y^3\pdy{3}u + \frac{1}{z}\pdx{2}\partial_zu $.

\noindent
Una superficie bidimensionale in $ \mathbb{R}^3 $ di equazione $ S(x,y,z) = 0 $ è caratteristica per l'equazione \eqref{eq:bestia} se vale che:
\begin{equation}
	(\px S)^3 + y^3(\py S)^3 + \frac{1}{z}(\px S)^2(\partial_z S\\) = 0
	\label{eq:caratteristica}
\end{equation}
Questa equazione si può risolvere con un ansatz di separazione: 
\begin{equation}
	 S(x,y,z) = X(x) + Y(y) + Z(z) 
	\label{eq:ansatz}
\end{equation}
Sostituendo \eqref{eq:ansatz} in \eqref{eq:caratteristica} si ottiene
\begin{equation}
\left(\frac{dX}{dx}\right)^3 + y^3 \left(\frac{dY}{dy}\right)^3 + \frac{1}{z}\left(\frac{dX}{dx}\right)^2\left(\frac{dZ}{dz}\right) = 0 
\label{eq:caratteristiche_separazione}
\end{equation} 
Questa equazione deve valore per ogni $ x,\,y $ e $ z $, quindi il secondo termine deve essere una costante $ A^3 $, da cui:
\[	y^3 \left(\frac{dY}{dy}\right)^3 = A^3 \Rightarrow \frac{dY}{dy} = \frac{A}{y} \Rightarrow Y(y) = a_Y \ln y	\]
Anche la somma degli altri due membri di \eqref{eq:caratteristiche_separazione} deve essere una costante $ B $:
\[ \left(\frac{dX}{dx}\right)^3 + \frac{1}{z}\left(\frac{dX}{dx}\right)^2\left(\frac{dZ}{dz}\right) = B \Rightarrow \left(\frac{dX}{dx}\right) + \frac{1}{z}\left(\frac{dZ}{dz}\right) = \frac{B}{\left(\frac{dX}{dx}\right)^2}	\]
Il secondo termine deve essere una costante $ C $:
\[ \frac{1}{z}\left(\frac{dZ}{dz}\right) = C \Rightarrow \left(\frac{dZ}{dz}\right) = Cz \Rightarrow Z(z) = a_Z z^2 + \mathrm{cost}	\]
Infine anche il restante termine di \eqref{eq:caratteristiche_separazione} deve essere costante:
\[ \left(\frac{dX}{dx}\right)^3 + C\,\left(\frac{dX}{dx}\right)^2= B  \]
Questa è un'equazione di terzo grado in $ X' $ e quindi avrà in generale tre soluzioni. Sia $ a_X $ una di queste:
\[ \left(\frac{dX}{dx}\right) = a_X \Rightarrow X(x) = a_X x + \mathrm{cost} \]
Quindi l'equazione della curva caratteristica è:
\[ S(x,y,z) = a_X x	+ a_Y \ln y + a_Z z^2 + \mathrm{cost}	\]
\qedsymbol
\section*{Esercizio 4}

\subsection*{Testo} Metodo delle caratteristiche

\begin{enumerate}
	\item Risolvere il problema quasilineare:
	\begin{equation}
	u\px u + \py u = 2, \quad u(x,x) = x,
	\label{eq:carat}
	\end{equation}
	Con il metodo delle caratteristiche (Lagrange-Charpit). 
	\item Come cambia la soluzione se prendiamo $ u(x,x) = \frac{x}{2} $ anziché $ u(x,x) = x $ per la curva iniziale?
\end{enumerate}
\subsection*{Svolgimento} 
\subsubsection*{Punto 1}
E'\ comodo introdurre il parametro $ \alpha $ per trattare contemporaneamente entrambi i punti dell'esercizio. L'equazione da risolvere è:
\[
\begin{cases}
u\px u + \py u - 2 = 0 \\
u(x,x) = \alpha x \qquad \alpha = \frac{1}{2},1 \\
\end{cases}
\]
Parametrizzando la curva iniziale con:
\[
\begin{cases}
x(0,s) = x_0(s) = s \\
y(0,s) = y_0(s) = s \\
u(0,s) = u_0(s) = \alpha s \\
\end{cases}
\]
Deve risultare che:
\[ \left|
\begin{array}{ccc}
	1 & \alpha s \\
	1 & 1 \\
\end{array} \right| \not= 0
\]
Ma questo determinante vale $ 1-\alpha s $ che non è identicamente nullo.
Le equazioni delle caratteristiche sono:
\[
\begin{cases}
	\dot{x} = u  \\
	\dot{y} = 1  \\
	\dot{u} = 2   
\end{cases}
\]
Applicando le condizioni a $ t=0 $ si ottiene:
  \begin{subequations}
  	\begin{numcases}{}
y = t + s \label{eq:equadit1} \\
u = 2t + \alpha s \\
x = t^2 + \alpha s t + s \label{eq:equadit3}
  \end{numcases}{}
    \end{subequations}
Sostituendo \eqref{eq:equadit1} in \eqref{eq:equadit3} si ottiene $ x = (1-\alpha)t^2 + (\alpha y -1) +y $.

\noindent Se $ \alpha = 1 $:
\[ t = \frac{x-y}{y-1} \qquad s = y - \frac{x-y}{y-1} \]
Ciò conduce immediatamente alla soluzione di \eqref{eq:carat}:
\[	u(x,y) = 2\frac{x-y}{y-1} + y - \frac{x-y}{y-1}	= \frac{y^2-2y+x}{y-1}\]
Si nota immediatamente che $ u(x,x) = x $ e con qualche calcolo in più si può verificare che questa risolve l'equazione \eqref{eq:carat}.
\subsubsection*{Punto 2}
Quando $ \alpha = \frac{1}{2} $ risulta impossibile invertire univocamente \eqref{eq:equadit3}. Si ottiene:

\[
\begin{cases}
t = \frac{2-y \pm \sqrt{(y-2)^2-8(y-x)}}{2} = \frac{2-y \pm \sqrt{y^2 -12y +8x +4}}{2} \\
s = y - \frac{2-y \pm \sqrt{(y-2)^2-8(y-x)}}{2} = y - \frac{2-y \pm \sqrt{y^2 -12y +8x +4}}{2}
\end{cases}
\]
Da cui si ricava la soluzione:
\begin{equation}	
u(x,y) = \frac{3}{2}t + \frac{1}{2}y = \frac{3}{2}- \frac{1}{4}y \pm 3\frac{\sqrt{y^2 -12y +8x +4}}{4} 
\label{eq:soluz_finale_2}
\end{equation}
Si verifica velocemente che $ u(x,x) = \frac{x}{2} $, la quale è condizione necessaria perché \eqref{eq:soluz_finale_2} sia la soluzione corretta.
\qedsymbol
\section*{Esercizio 5}

\subsection*{Testo} L'equazione di Burgers

\begin{enumerate}
	\item Risolvere l'equazione di Burgers inviscida con il metodo di Lagrange-Charpit prendendo come condizione iniziale $ u(x,0) = u_0(x) $
	\begin{equation}
	\pt u + u \px u = 0
	\label{eq:burgers2}
	\end{equation}
	\item Si faccia vedere che la trasformazione di Hopf-Cole:
	\begin{equation}
	u\xt = -2 \nu \px \ln \psi\xt
	\label{eq:hopf}
	\end{equation}	
	trasforma l'equazione di Burgers viscosa	
	\begin{equation}
 	\pt u + u \px u = \nu \pdx{2}u 
	\label{eq:burgers3}
	\end{equation}	 
	nell'equazione del calore:
	\begin{equation}
	\pt \psi\xt = \nu \pdx{2} \psi \xt
	\label{eq:cal}
	\end{equation}
	Trasformare la condizione iniziale $ u(x,0) $ in una condizione iniziale per $ \psi $ e risolvere la \eqref{eq:cal} usando il nucleo di calore.	
\end{enumerate}

\subsection*{Svolgimento} 
\subsubsection*{Punto 1}
Per comodità nel seguente svolgimento si chiama la variabile $ t $ come $ y $ e a partire da \eqref{eq:burgers2} si definiscono $ a(x,y,u) = u,\, b(x,y,u) = 1,\,c(x,y,u)= 0 $.
Parametrizzando la curva su cui sono definiti i dati iniziali con il parametro $ s \in \mathbb{R} $:
\[ x_0(s) = s \qquad y_0(s) = 0 \qquad u_0(s) = u_0(s)	\]
Si verifica che questa non è una curva caratteristica:
\[ \left|
\begin{array}{ccc}
1 & u_0(s) \\
0 & 1 \\
\end{array} \right| = 1 \not= 0 \]
Le equazioni delle caratteristiche sono (dove il punto indica la derivata rispetto a $ t $):
\[
\begin{cases}
\dot{x} = u  \\
\dot{y} = 1  \\
\dot{u} = 0 
\end{cases}
\]
Dalla seconda si ottiene:
\[	\dot{y} = 1 \Rightarrow y(t,s) = t + \mathrm{cost} \text{ ma } y(0,s) = 0 \text{ quindi } y(t,s) = t	\]
Dalla terza:
\[	\dot{u} = 0 \Rightarrow u(t,s) = \mathrm{cost} \text{ ma } u(0,s) = u_0(s)	\text{ quindi } u(t,s) = u_0(s)	\]
Infine dalla prima:
\[	\dot{x} = u \Rightarrow x(t,s) = u_0(s)t + \mathrm{cost} \text{ ma } x(0,s) = s \text{ quindi } x(t,s) = u_0(s)t + s	\]
Questa ultima definisce implicitamente la relazione tra $ x $ e $ s $. La soluzione di \eqref{eq:burgers2} può essere quindi scritta implicitamente come:
\[	u(s,0) = u(y) = u(s + u_0(s)y, y)		\]
Rinominando la variabile $ y $ in $ t $:
\[	u(s,0) = u(t) = u(s + u_0(s)t, t)		\]
\subsubsection*{Punto 2}
Manipolando l'equazione \eqref{eq:burgers3}: $ \pt u + u \px u = \nu \pdx{2}u  \Rightarrow \pt u = \nu \pdx{2}u -u\px u = \px (\nu \px u - \frac{u^2}{2}) $ e applicando la sostituzione:
\[ \pt (-2\nu \px \ln\psi)	= \px \left(\nu \px (-2\nu \px \ln\psi)- \frac{(-2\nu\px\ln\psi)^2}{2}\right)	\]
Assumendo sufficiente regolarità si possono scambiare le derivate temporali e spaziali al primo membro e integrare ottenendo:
\[ -2\nu \pt \ln\psi	= \nu \px (-2\nu \px \ln\psi) - \frac{(-2\nu\px\ln\psi)^2}{2} +c(t)	\]
Svolgendo le derivate e considerando che $ \px\ln\psi = \frac{\px\psi}{\psi} $ si giunge a:
\[\frac{\pt\psi}{\psi} = \nu \frac{\pdx{2}\psi}{\psi} + \tilde{c}(t)	\]
Da cui si arriva all'equazione del calore assumendo di poter porre $ \tilde{c}(t)\psi = 0 $.
La condizione iniziale $ u(x,0) $ trasforma così:
\[ u(x,0) = -2\nu \px\ln\psi(x,0) \Rightarrow \px\ln\psi = \frac{u(x,0)}{-2\nu} \Rightarrow\ln\psi(x,0) = \int\frac{u(x,0)}{-2\nu}	\]
Quindi:
\[ \psi(x,0) = \exp\left(\int\frac{u(x,0)}{-2\nu}\right)	\]
A questo punto si può risolvere \eqref{eq:cal} utilizzando il nucleo di calore con questa condizione iniziale:
\[	\psi(x,t) = \frac{1}{\sqrt{4\pi\nu t}}\int\exp\left(-\frac{(x-y)^2}{4\nu t} - \int\frac{u(z,0)}{2\nu}dz \right)dy	\]
\qedsymbol

\section*{Esercizio 6}

\subsection*{Testo} Effettuare le trasformazioni di B\"acklund sull'equazione di Sine-Gordon
\begin{equation}
\px \pt u = \sin u
\label{eq:sine_gordon}
\end{equation}
con le trasformazioni:
  \begin{subequations}
  	\begin{numcases}{}
\px  u + \px v = 2 a \sin{\frac{u-v}{2}}  \label{eq:back1} \\
\pt  u - \pt v = \frac{2}{a} \sin{\frac{u+v}{2}} \label{eq:back2}
		\end{numcases}
	\end{subequations}

\subsection*{Svolgimento}
Derivando \eqref{eq:back1} per $ t $ e \eqref{eq:back2} per $ x $ si ottiene:
  \begin{subequations}
		\begin{numcases}{}
			\pt\px u + \pt\px v = 2a \pt \sin{\frac{u-v}{2}} = a \cos{\frac{u-v}{2}} (\pt u - \pt v)\\
			\px\pt u - \px\pt v = \frac{2}{a} \px \sin{\frac{u-v}{2}} = \frac{1}{a} \cos{\frac{u+v}{2}} (\px u + \px v) 
		\end{numcases}
  \end{subequations}
Utilizzando le trasformazioni di B\"acklund sui membri destri di queste equazioni si ottiene:
\[
\begin{cases}
\pt\px u + \pt\px v = 2 \cos{\frac{u-v}{2}} \sin{\frac{u+v}{2}} \\
\px\pt u - \px\pt v = 2 \cos{\frac{u+v}{2}} \sin{\frac{u-v}{2}} \\
\end{cases}
\]
Sottraendo la prima equazione dalla seconda si giunge a:
\[	\pt\px v = \cos{\frac{u-v}{2}} \sin{\frac{u+v}{2}}  - \cos{\frac{u+v}{2}} \sin{\frac{u-v}{2}}	\]
Ricordando l'elementare relazione goniometrica $ \sin{\left(\alpha - \beta\right)} = \sin{\alpha}\cos{\beta} - \cos{\alpha}\sin{\beta} $ si ritorna alla forma \eqref{eq:sine_gordon}:
\[	\px \pt v = \sin (\frac{u + v}{2} - \frac{u + v}{2}) = \sin v	\]
Una soluzione elementare di questa equazione è $ v\xt = 0 $, con la quale le trasformazioni di B\"acklund diventano:
\[
  	\begin{cases}
  	\px  u = 2 a \sin{\frac{u}{2}} \\
  	\pt  u = \frac{2}{a} \sin{\frac{u}{2}}
  	\end{cases}
\]
Cioè definendo $ w(x,y) = \frac{u(x,y)}{2} $:
  \begin{subequations}
  	\begin{numcases}{}
  	\px  w =  a \sin{w}  \label{eq:back1w} \\
  	\pt  w = \frac{1}{a} \sin{w} \label{eq:back2w}
  	\end{numcases}
  \end{subequations}
Questo è un sistema di equazioni differenziali alle derivate parziali non lineari con una sola funzione incognita. Moltiplicando \eqref{eq:back1w} per $ dx $ e \eqref{eq:back2w} per $ dt $ si giunge a:
  \begin{subequations}
  	\begin{numcases}{}
  	\px  w\,dx =  a \sin{w}\,dx \label{eq:back1wdiff} \\
  	\pt  w\,dt = \frac{1}{a} \sin{w}\,dt\label{eq:back2wdiff}
  	\end{numcases}
  \end{subequations}
Sommando \eqref{eq:back1wdiff} con \eqref{eq:back2wdiff} e osservando che $ dw = \px w dx + \pt w dt $ si ottiene:
\[	dw = a\sin w\,dx + \frac{1}{a} \sin w\,dt \quad	\Rightarrow  \quad \frac{dw}{\sin w} = a\,dx + \frac{dt}{a}\]
Questa è una forma differenziale integrabile:
\[
\begin{split}
\int \frac{dw}{\sin w} & = \int \frac{2i}{\mathrm{e}^{iw}-\mathrm{e}^{-iw}}dw = \int \frac{2i\,\mathrm{e}^{iw}}{\mathrm{e}^{2iw}-1}dw \stackrel{z=\mathrm{e}^{iw}}{=} \int \frac{2}{z^2-1}dz \\ & = \ln\left(z-1\right) - \ln\left(z-1\right) + \mathrm{cost} = \ln\left( \frac{z-1}{z+1} \right) + \mathrm{cost} = \\ & = \ln\left( \frac{\mathrm{e}^{iw}-1}{\mathrm{e}^{iw}+1} \right) + \mathrm{cost} = \ln\left(\tan\frac{w}{2}\right)	+ \mathrm{cost}
\end{split}
\]
Quindi, se $ \delta $ è la costante di integrazione:
\[	\ln\left(\tan\frac{w}{2}\right) = ax + \frac{t}{a} + \delta 	\]
Da cui, ricordando che $ w(x,t) = \frac{u(x,t)}{2} $:
\[	u(x,t) = 4 \arctan\left(\mathrm{e}^{ax + \frac{t}{a} + \delta}\right)		\]
Questa è una soluzione non banale dell'equazione \eqref{eq:sine_gordon}. 
\qedsymbol
\end{document}          

