\documentclass[]{scrartcl}
\usepackage{lmodern}
\usepackage[italian]{babel}
\usepackage[utf8]{inputenc}
\usepackage{amsmath, amsfonts, amssymb, amsthm}
\usepackage[makeroom]{cancel}
\usepackage{cases}
\usepackage{braket}
\usepackage{siunitx}
\usepackage{tikz, pgfplots}

\newtheorem{lemma}{Lemma}

\usetikzlibrary{decorations.pathmorphing}
\usepgfplotslibrary{fillbetween}
\usetikzlibrary{arrows.meta}

\tikzset{snake it/.style={decorate, decoration={snake, amplitude = 1.25}}}

\usepackage{physics}

\renewcommand{\qedsymbol}{\hfill \rule{0.5em}{0.5em}}
\newcommand{\me}{\mathrm{e}}
\newcommand{\pd}{\partial}
\newcommand{\pfrac}[1]{\frac{\partial}{\partial #1}}
\newcommand{\cost}{\mathrm{cost}}
\newcommand{\sg}{\sqrt{-g}}
\newcommand{\hg}{\hat{g}}
\renewcommand{\d}[1]{\ensuremath{\operatorname{d}\!{#1}}}
\renewcommand{\dd}[1]{\ensuremath{\operatorname{d}^2\!{#1}}}
\newcommand{\dddd}[1]{\ensuremath{\operatorname{d}^4\!{#1}}}
\newcommand{\hF}{{}^\star F}
\newcommand{\htF}{{}^\star \tilde{F}}

\title{Esercizi di Gravità e Superstringhe I}
\author{Gabriele Bozzola \\ Matricola: 882709}
\date{Settembre 2016}

\begin{document}
\maketitle

\section*{Esercizio 1}

\subsection*{Testo}

Si consideri un modello di inflazione con un singolo inflatone $ \phi $ accoppiato
minimalmente al campo gravitazionale, e con un potenziale quadratico $ V(\phi) = \mu^2 \phi^2 $.
Misuriamo i tempi con il tempo proprio $ \tau $ degli osservatori isotropi e chiamiamo $ \phi_i $
il valore dell'inflatone al tempo iniziale $ \tau_i $ dell'inflazione, e $ \phi_f $ il suo valore
finale al termine $ \tau_f $ dell'epoca inflazionaria.
Indichiamo la durata dell'inflazione con $ \Delta \tau = \tau_f - \tau_i $.
\begin{enumerate}
\item Trovare i parametri di \emph{slow-roll} $ \epsilon $ e $ \eta $ per questo potenziale.
  Quale è la condizione che deve soddisfare l'inflatone $ \phi $ affinché valga l'approssimazione
  di \emph{slow-roll}?
\item Per quale valore $ \phi_f $ l'evoluzione esce dal regime di \emph{slow-roll} e inizia la fase
  di \emph{reheating}?
\item Esprimere il numero di e-foldings $ N = \ln(a_f\over a_i) $ come funzione di $ \phi_f $ e di
  $ \phi_i $. Come dipende il valore iniziale $ \phi_i $ dell'inflatone da $ N $? Quanto vale,
  in unità di $ M_P $, per $ N \approx 60 $?
\item Risolvere il sistema di equazioni per l'evoluzione del fattore di scala $ a(\tau) $ e
  dell'inflatone $ \phi(\tau) $ in regime di \emph{slow-roll}, e trovare la dipendenza esplicita
  del fattore di scala $ a(\tau) $ dal tempo cosmico $ \tau $.
\item Come dipende la durata $ \Delta \tau $ dell'inflazione da $ N $ e $ \mu $? Calcolare $ \Delta\tau $
  in secondi, se l'inflatone ha massa del bosone di Higgs.
\item \'E corretto assimilare l'inflazione con una fase di espansione esponenziale di tipo de-Sitter?
  Su che scale di tempi si verifica una deviazione dell'evoluzione quasi de-Sitter?
\end{enumerate}

\subsection*{Soluzione}

In $ 3 + 1 $ dimensioni l'azione di Einstein-Hilbert accoppiata con un campo scalare reale $ \phi $ soggetto
ad un potenziale $ V(\phi) $ è:
\[
  S =  \int \dddd x  \sg \left( \frac{M_P^2}{2} R  - \frac{1}{2}\left( \nabla\phi \right)^2 + V(\phi) \right)
\]
con $ g $ determinante della metrica, $ R $ curvatura scalare e  $ M_P $ massa di Planck ridotta, definita da $ M_P = \frac{1}{8 \pi G_N} $.
Le equazioni del moto di questa azione sono:
\begin{numcases}{}
  \ddot{\phi} + 3 H \dot{\phi} + V'(\phi) = 0  \label{eq:eq_moto_campo_scalare} \\
  H^2 = \frac{1}{3 M_P^2} \left( \frac{1}{2} \dot{\phi}^2 + V(\phi) \right)  \label{eq:eq_friedmann}
\end{numcases}
con $ H $ parametro di Hubble definito da $ H = \dot{a} \slash a $.
L'equazione \eqref{eq:eq_moto_campo_scalare} proviene dalla variazione dell'azione
rispetto $ \phi $ mentre l'equazione \eqref{eq:eq_friedmann} rispetto $ g^{\mu\nu} $.
Specializzando queste equazioni al potenziale dell'esercizio, il quale, tra l'altro, si nota essere sempre non negativo:
\begin{numcases}{}
  \ddot{\phi} + 3 H \phi + 2 \mu^2 \phi = 0  \label{eq:eq_moto_campo_scalare_esercizio} \\
  H^2 = \frac{1}{3 M_P^2} \left( \frac{1}{2} \dot{\phi}^2 + \mu^2 \phi^2 \right)  \label{eq:eq_friedmann_esercizio}
\end{numcases}
Affinché il campo inflatonico produca gli effetti di una costante cosmologica, quindi causi l'espansione accelerata,
riducendo quindi il raggio di Hubble e risolvendo il problema della piattezza, devo poter trattare il campo come un
fluido perfetto di equazione di stato $ P = - \rho $.
Il tensore energia-impulso dell'inflatone si dimostra essere:
\[
  T_{\mu\nu} = \frac{-2}{\sg}\frac{\delta \mathcal{L}_\phi}{\delta g^{\mu\nu}} \quad \rightarrow \quad
  T_{\tau\tau} = \frac{1}{2}\dot{\phi}^2 + V(\phi) =: \rho \qquad T_{\chi\chi} = a^2 \left( \frac{1}{2}\dot{\phi}^2 - V(\phi) \right) =: a^2 P
\]
Questo non è il tensore energia-impulso di un fluido perfetto perché non vi è un'equazione di stato algebrica che lega pressione e densità, tuttavia se $ \dot{\phi}^2 \ll V(\phi) $ allora $ \rho \approx V(\phi) $ e $ P \approx - V(\phi) $,
in questo modo si ha il medesimo comportamento di una costante cosmologica, cioè un fluido perfetto con equazione
di stato $ P = - \rho $ e quindi la seconda equazione di Friedman diventa:
\[
  \frac{\ddot{a}}{a} = -\frac{1}{6 M_P^2} \left( \rho + 3P \right) = - \frac{1}{3 M_P^2} \left( \dot{\phi}^2 - V(\phi) \right)
  \approx \frac{V(\phi)}{3 M_P^2}
\]
Questa determina un'espansione accelerata.
Affinché l'espansione duri a sufficienza il moto deve essere sovrasmorzato, cioè deve valere $ \ddot{\phi} \ll 3 H \phi $,
senza questa richiesta il campo oscillerebbe intorno al minimo del potenziale e quindi dominerebbe l'energia cinetica,
invalidando la prima assunzione. Le richieste per entrare nel regime di \emph{slow-roll} sono quindi:
\begin{numcases}{}
  \frac{1}{2} \dot{\phi{}} \ll \mu^2 \phi^2 \\
  \ddot{\phi} \ll 3 H \phi
\end{numcases}
Queste possono essere riscritte in termini dei parametri di \emph{slow-roll} $ \epsilon $ e $ \eta $:
\begin{align}
  \epsilon = \frac{M_P^2}{2}\left( \frac{V'}{V} \right)^2 \approx \frac{M_P^2}{\phi^2} \label{eq:espilon} \\
  \eta = M_P^2 \frac{V''}{V} \approx \frac{M_P^2}{\phi^2} \label{eq:eta}
\end{align}
Nel quale ho soppresso fattori moltiplicativi dell'ordine dell'unità.
Le condizioni di \emph{slow-roll} diventano quindi:
\begin{numcases}{}
  \varepsilon \ll 1 \quad \rightarrow \quad \abs{\phi} \gg M_P \\
  \eta \ll 1 \quad \rightarrow \quad \abs{\phi} \gg M_P
\end{numcases}
Condizione necessaria per il regime di \emph{slow-roll} è perciò $ \abs{\phi} \gg M_P $
\footnote{Questa non è una condizione sufficiente per entrare in un regime inflazionario:
  perché si abbia inflazione basta che valga $ \frac{P}{\rho} = w < -\frac{1}{3} \Leftrightarrow \frac{\dot{\phi}^2 + V(\phi)}{\dot{\phi}^2 - V(\phi)} < -\frac{1}{3} $,
la quale è automaticamente soddisfatta imponendo le condizioni di \emph{slow-roll}.}.
Qualora questa condizione smette di essere verificata si esce dal regime di \emph{slow-roll} e si inizia il \emph{reheating}, che avviene quindi
per $ \abs{\phi(\tau_f)} \approx M_P $. \\ \\
Se $ N = \ln \left( \frac{a(\tau_f)}{a(\tau_i)} \right) $ è il numero di e-folds, allora:
\[
  N = \ln \left( \frac{a(\tau_f)}{a(\tau_i)} \right) = \int_{a(\tau_i)}^{a(\tau_f)} \frac{da}{a} =
  \int_{\tau_i}^{\tau_f}\frac{\dot{a}}{a} \d{\tau} = \int_{\tau_i}^{\tau_f} H \d{\tau} = \int_{\tau_i}^{\tau_f} \frac{H^2}{H} \d{\tau}
\]
Utilizzando le equazioni di campo in approssimazione di \emph{slow-roll} riportate tra poche righe (equazioni \eqref{eq:eq_moto_campo_scalare_sr}
e \eqref{eq:eq_friedmann_sr}):
\[
  \int_{\tau_i}^{\tau_f} \frac{H^2}{H} \d{\tau} = - \int_{\tau_i}^{\tau_f} \frac{\mu^2 \phi^2}{3 M_P^2} \frac{3 \dot{\phi}}{2 \mu^2 \phi} \d{\tau}
  = - \frac{1}{2 M_P^2} \int_{\tau_i}^{\tau_f} \phi \dot{\phi} \d{\tau}
\]
Cambiando variabile di integrazione da $ \tau $ a $ \phi $ e assumendo che $ \phi(\tau_f) \approx M_P $:
\[
  - \frac{1}{2 M_P^2} \int_{\tau_i}^{\tau_f} \phi \dot{\phi} \, \d{\tau} \approx - \frac{1}{2 M_P^2} \int_{\phi(\tau_i)}^{M_P} \phi \, \d \phi
  = \frac{\phi^2(\tau_i) - M_P^2 }{4 M_P^2} \overset{\phi^2 \gg M_P^2}{\approx} \frac{\phi^2(\tau_i)}{4 M_P^2}
\]
Affinché sia $ N \approx 60 $ deve essere $ \phi(\tau_i) = \sqrt{4 N} M_P \approx \sqrt{240} M_P \approx \num{15.5} M_P $. \\ \\
Le equazioni di campo in regime di \emph{slow-roll} sono:
\begin{numcases}{}
  3 H \dot{\phi} = - V'(\phi)  \quad \rightarrow \quad 3 H \dot{\phi} = - 2 \mu^2 \phi \label{eq:eq_moto_campo_scalare_sr} \\
  H^2 = \frac{V(\phi)}{3 M_P^2}  \quad \rightarrow \quad H^2 = \frac{\mu^2 \phi^2}{3 M_P^2}  \label{eq:eq_friedmann_sr}
\end{numcases}
Dato che $ H = \dot{a} \slash a = \frac{d}{\d{\tau}} \ln a$ questo è un sistema di equazioni differenziali nelle funzioni $ a(\tau) $ e $ \phi(\tau) $.
Elevo \eqref{eq:eq_moto_campo_scalare_sr} al quadrato e divido per \eqref{eq:eq_friedmann_sr}:
\[
  9 H^2 \dot{\phi}^2 = 4 \mu^4 \phi^2 \quad \xrightarrow{\text{Divisa per \eqref{eq:eq_friedmann_sr}}} \quad
  3 \dot{\phi}^2 = 4 \mu^2 M_P^2
\]
Assumendo il campo positivo e la sua derivata negativa:
\[
  \dot{\phi} = - \frac{2}{\sqrt{3}}M_P \mu \quad \rightarrow \quad \phi(\tau) = \phi(\tau_i) - \frac{2}{\sqrt{3}}M_P \mu \left( \tau - \tau_i \right)
\]
Sostituendo in \eqref{eq:eq_moto_campo_scalare_sr} si ottiene:
\[
  H = \frac{\sqrt{3}}{3}\frac{\mu}{M_P}\phi(\tau_i) - \frac{2}{3}\mu^2 (\tau - \tau_i) \quad \rightarrow \quad  \frac{d}{\d{\tau}} \ln a (\tau) =
  \frac{\sqrt{3}}{3}\frac{\mu}{M_P}\phi(\tau_i) - \frac{2}{3}\mu^2 (\tau - \tau_i )
\]
Rinominando le costanti di integrazione in modo da mettere in evidenza il valore iniziale del fattore di scala:
\[
  a(\tau) = a(\tau_i) \, \me^{\, \phi(\tau_i) (\tau - \tau_i)- \frac{\sqrt{3}}{3} M_P \mu (\tau - \tau_i)^2 }
\]
Si può esplicitare la dipendenza della durata dell'inflazione dal numero di e-foldings, da $ \mu $ e dal valore iniziale dell'inflatone:
\[
  N = \ln \left( \frac{a(\tau_f)}{a(\tau_i)} \right) =  \phi(\tau_i) (\tau_f - \tau_i)- \frac{\sqrt{3}}{3} M_P \mu (\tau_f - \tau_i)^2
\]
Ponendo il valore iniziale del campo $ \phi(\tau_i) = 2 \sqrt{N} M_P $ e approssimando $ \frac{\sqrt{3}}{3} \approx 1 $
si ottiene l'equazione di secondo grado:
\begin{equation}
  \label{eq:tempo_inflazione}
  M_P \mu \left( \Delta \tau \right)^2 - 2 \sqrt{N} M_P \Delta \tau + N = 0 \quad
  \rightarrow \quad
  \left( \Delta \tau \right)^2 - \frac{2 \sqrt{N}}{\mu} \Delta \tau + \frac{N}{M_P \mu} = 0
\end{equation}
Alcune osservazioni dimensionali: nel sistema di unità di misura di Planck l'unica dimensione rilevante è la massa e $ \hbar = 1 $, $ c = 1 $
e $ G = 1 $. I tempi hanno dimensione in massa $ -1 $, in quanto $ \hbar = [E][T] $, ma $ [E] = [M] $ dato che $ E = mc^2 $, e siccome $ \hbar $
non ha dimensioni $ [T] = [M]^{-1} $. L'equazione \eqref{eq:tempo_inflazione} risulta quindi avere le dimensioni corrette: $ M_P $ e $ \mu $ hanno
dimensioni in massa $ 1 $, mentre $ \Delta \tau $ ha dimensioni $ -1 $ quindi tutti i termini sono adimensionali omogenei.
Risolvendo l'equazione \eqref{eq:tempo_inflazione}:
\[
  \Delta \tau = \frac{\sqrt{N}}{\mu} \pm \sqrt{\frac{N}{\mu^2} - \frac{N}{M_P\mu}}
\]
Siccome $ M_P \gg \mu $ è possibile trascurare il secondo termine sotto la radice rispetto al primo, in questo modo ottengo due risultati
$ \Delta \tau = 0 $ e $ \Delta \tau = \frac{\sqrt{N}}{\mu} $.
Se esprimo $ \mu $ in $ \si{\GeV} $ ottengo $ \Delta \tau $ con dimensioni $ [E]^{-1} $, ma nelle unità di Planck posso moltiplicare il numeratore
per $ \hbar = 1 $, che nelle consuete unità di misura ha dimensioni di $ [E][T] $, ripristinando così le giuste dimensioni di tempo:
\[
  \Delta \tau = \frac{\sqrt{60}\,\hbar}{\SI{125}{\GeV}}\approx \SI{E-26}{\second}
\]
Se non avessi trascurato il secondo termine, la soluzione con il segno negativo sarebbe stata inferiore al tempo di Planck, e quindi inaccettabile.
% Considerando che la soluzione di de-Sitter può essere parametrizzata con coordinate inflazionarie in modo da prendere una forma alla FLRW in cui è messo
% in evidenza in fattore di scala:
% \[
  % ds^2 = -dt^2 + \me^{2 t \slash l}d\vec{x}^2
  % \]
\\ \\
L'inflazione è assimilabile ad un'espansione del tipo de-Sitter, solo per piccoli tempi, quando vale $ \left( \Delta \tau \right)^2
\ll \Delta \tau $, quando la crescita è esponenziale. % per tempi maggiori l'evoluzione è mal descritta da un'espansione di de-Sitter.

\section*{Esercizio 2}

\subsection*{Testo}
Si consideri la varietà quadridimensionale $ \mathcal{M} $ dotata della metrica:

\begin{equation}
  \label{eq:metrica_problema_2}
  ds^2 = \frac{8 m w^2}{2 m + v w} \d v ^2 + 8 m \d v  \d w + \left( 2m + v w \right)^2 \left( \d \theta^2 + \sin^2 \theta \d \varphi^2 \right)
\end{equation}
dove $ m > 0 $ è una costante. Si mostra che questa metrica è una soluzione delle equazioni di Einstein nel vuoto,
ha un singolarità di curvatura in $ vw = - 2 m $, ed è regolare per $ vw > -2m $.
\begin{enumerate}
\item Mostrare che l'applicazione $ \phi_\lambda $ da $ \mathbb{R} \cross \mathcal{M} $ in $ \mathcal{M} $ tale che,
  fissato $ \lambda \in \mathbb{R} $, sia data da:
  \begin{align*}
    \label{eq:phi_lambda}
    \phi_\lambda : & \quad \mathcal{M} \rightarrow \mathcal{M} \\
                   & \quad \left( v,w,\theta,\varphi \right) \mapsto \left( \me^\frac{\lambda}{4m} v, \me^{-\frac{\lambda}{4m}} w, \theta, \varphi  \right)
  \end{align*}
  è un gruppo a un parametro di isometrie per la metrica \eqref{eq:metrica_problema_2}, e che il campo vettoriale
  che lo genera è dato da
  \begin{equation}
    \label{eq:campo_vettoriale}
    \xi = \frac{1}{4 m} \left( v \pfrac{v} - w \pfrac{w} \right)
  \end{equation}
  Dedurre che $ \xi $ è un campo vettoriale di Killing per la metrica \eqref{eq:metrica_problema_2}.
\item Si consideri la famiglia di ipersuperfici $ \set{ w = \cost } $. Trovare il generico campo
  vettoriale $ n^\mu $ normale a queste ipersuperfici. Mostrare che $ \mathcal{N} = \set{ w = 0} $ è un'ipersuperficie nulla.
\item Mostrare che $ \mathcal{N} $ è un'orizzonte di Killing per il campo vettoriale di Killing $ \xi $, e calcolare la
  gravità superficiale.
\item Sia $ \mathcal{A}(v,w) $ l'area delle sfere con $ (v,w) $ costante. Definire una coordinata di Schwarzschild
  $ r = \sqrt{\frac{\mathcal{A}(v,w)}{4\pi}} $ per la metrica \eqref{eq:metrica_problema_2} e passare alle coordinate
  $ (v,r,\theta,\varphi) $. Trovare una riparametrizzazione $ v = v(u) $ della coordinata $ v $ tale che gli elementi metrica
  \eqref{eq:metrica_problema_2} in coordinate $ (u, r, \theta, \varphi) $ siano indipendenti da $ u $. Riconoscete la metrica
  in queste coordinate? Potevate aspettarvelo prima di effettuare il cambiamento di coordinate?
\item Esprimere il campo vettoriale $ \xi $ nelle coordinate $ (u,r,\theta,\varphi) $ e confermare così che si tratta di un
  campo vettoriale di Killing.
\item Tracciare l'analogo del diagramma di Krustal per la metrica \eqref{eq:metrica_problema_2} in coordinate $ (u,w,\theta,\varphi) $
  evidenziando le curve con $ r $ costante, la singolarità di curvatura, l'orizzonte di Killing di $ \xi $, e le orbite del campo
  vettoriale di Killing $ \xi $. Di che natura (spazio, tempo, o luce) sono le rette con $ v $ costante e quelle con $ w $ costante?
  Quale è il vantaggio del diagramma di Kruskal rispetto al diagramma in coordinate $ (v,w) $?
\end{enumerate}

\subsection*{Soluzione}
Per mostrare che $ \phi_\lambda $ è un gruppo a un parametro di isometrie verifico inizialmente che queste trasformazioni formano un
gruppo, cioè soddisfano:
\begin{enumerate}
\item $ \phi_\lambda \circ \phi_\Lambda = \phi_{\lambda + \Lambda} \quad \forall \lambda,\Lambda \in \mathbb{R}$
\item $ \phi_0  = \mathbb{I} $.
\end{enumerate}
Ma:
\begin{enumerate}
\item Applicando le due trasformazioni:
\[
  (v,w,\theta,\varphi) \overset{\phi_\lambda}{\mapsto} \left( \me^\frac{\lambda}{4m} v, \me^{-\frac{\lambda}{4m}} w, \theta, \varphi  \right)
  \overset{\phi_\Lambda}{\mapsto} \left( \me^\frac{\lambda + \Lambda}{4m} v, \me^{-\frac{(\lambda + \Lambda)}{4m}} w, \theta, \varphi  \right)
\]
il quale ovviamente corrisponde all'azione di $ \phi_{\lambda + \Lambda} $ sul punto di partenza.
\item Triviale.
\end{enumerate}
Verifico ora che per ogni $ \lambda $ reale la trasformazione $ \phi_\lambda $ è un'isometria verificando che $ \phi_\lambda $ manda $ \d s^2 $ in
$ \d s^2 $, per far ciò noto che:
\begin{align*}
  \d v  & \mapsto \me^\frac{\lambda}{4m}\, \d v  \\
  \d w & \mapsto \me^{-\frac{\lambda}{4m}}\, \d w \\
  \d \theta & \mapsto \d \theta \\
  \d \varphi & \mapsto \d \varphi
\end{align*}
Da cui:
\begin{align*}
  \d v \,\d w & \mapsto \d v \,\d w \\
  vw & \mapsto vw \\
  w\,\d v  & \mapsto w\,\d v
\end{align*}
Per cui risulta che $ \d s^2 \mapsto \d s^2 \, \, \forall \lambda \in \mathbb{R} $, quindi le trasformazioni sono isometrie.
Considero ora l'azione di $ \me^{\lambda \xi} $: % sul generico punto $ (v,w,\theta,\varphi) $
\[
  \me^{\lambda \xi} = \sum_{k = 0}^{+\infty} \frac{\lambda^k}{k!}\xi^k = \sum_{k = 0}^{+\infty} \frac{\lambda^k}{k!} \left(\frac{1}{4 m} \right)^k
  \left( v \pfrac{v} - w \pfrac{w} \right)^k
\]
Sul generico punto $ (v,w,\theta,\varphi) $ l'operatore differenziale $ \left( v \pfrac{v} - w \pfrac{w} \right)^k $ agisce come l'identità
$ \forall k \in \mathbb{N} $, quindi:
\[
  \me^{\lambda \xi} : \left( v,w,\theta,\varphi \right) \mapsto \left( \me^\frac{\lambda}{4m} v, \me^{-\frac{\lambda}{4m}} w, \theta, \varphi  \right)
\]
% nel caso $ \lambda $ sia infinitesimo, e risulta
% quindi possibile approssimare $ \me^{\lambda \xi} \approx 1 + \lambda \xi $:
% \[
   % \me^{\lambda \xi} \approx 1 +  \frac{\lambda}{4 m} \left( v \pfrac{v} - w \pfrac{w} \right) :  (v,w,\theta,\varphi) \mapsto
   % (v + \frac{\lambda}{4 m} v, w - \frac{\lambda}{4 m} w, \theta, \varphi )
   % \approx \left( \me^\frac{\lambda}{4m} v, \me^{-\frac{\lambda}{4m}} w, \theta, \varphi  \right)
% \]
Questa è esattamente la trasformazione $ \phi_\lambda $. Siccome $ \xi $ genera un gruppo a un parametro di isometrie allora è un vettore di Killing. \\ \\
Il tensore metrico per \eqref{eq:metrica_problema_2} è:
\[
  g_{\mu\nu} =
  \begin{pmatrix}
    \frac{8 m w^2}{2m + vw} & 4m                                & 0                           & 0 \\
    4m                      & 0                                 & 0                           & 0 \\
    0                       & 0                                 & \left( 2m + vw \right)^2    & 0 \\
    0                       & 0                                 & 0                           & \left( 2m + vw \right)^2 \sin^2 \theta
  \end{pmatrix}
\]
Il suo inverso è il tensore:
\[
  g^{\mu\nu} =
  \begin{pmatrix}
    0                       & \frac{1}{4m}                      & 0                           & 0 \\
    \frac{1}{4m}            & - \frac{w^2}{4m^2 + 2mvw} & 0                           & 0 \\
    0                       & 0                                 & \left( 2m + vw \right)^{-2} & 0 \\
    0                       & 0                                 & 0                           & \left( 2m + vw \right)^{-2} \sin^{-2} \theta
  \end{pmatrix}
\]
Il generico vettore normale alle ipersuperfici di equazione $ K(v,w,\theta,\varphi) = 0 $ ha la forma:
\begin{equation}
  \label{eq:vettore_normale}
  n^\mu(v,w,\theta,\varphi) = h(v,w,\theta,\varphi) g^{\mu\nu}\nabla_\nu K
\end{equation}
Nel caso considerato $ K(v,w,\theta,\varphi) = w $ quindi l'unico contributo proviene da $ \nu = w $, cioè:
\[
  n^\mu(v,w,\theta,\varphi) = h(v,w,\theta,\varphi) g^{\mu w} %\pfrac{w}
\]
Da cui:
\[
  n(v,w,\theta,\varphi) = h(v,w,\theta,\varphi) \left( \frac{1}{4 m} \pfrac{v} - \frac{1}{2m}\frac{w^2}{2m + vw} \pfrac{w}  \right)
\]
Ridefinendo opportunamente la funzione di normalizzazione $ h $:
\[
  n(v,w,\theta,\varphi) = h(v,w,\theta,\varphi) \left( \pfrac{v} - \frac{2w^2}{2m + vw} \pfrac{w}  \right)
\]
Il modulo di questo vettore è:
\begin{align*}
  n^{\mu}n_{\mu} &=  n^{\mu}n^{\nu}g_{\mu\nu} = n^v n^v g_{vv} + 2 n^v n^w g_{vw} + n^w n^w g_{ww} = \\
                 &= h^2 \frac{8mw^2}{2m + vw} - h^2 \frac{16mw^2}{2m + vw} = -h^2 \frac{8mw^2}{2m + vw}
\end{align*}
Tale modulo si annulla per $ w = 0 $, per questo $ \mathcal{N} $ è un'ipersuperficie nulla.
Inoltre siccome $ h^2 \geq 0 $, $ m > 0 $, $ 2m + vw > 0 $ e $ w^2 \geq 0 $ si nota che il vettore normale
ha sempre modulo negativo (ad eccezione di $ w = 0 $), quindi le superfici a $ w = \cost $ sono tutte di tipo spazio.
Il vettore di killing $ \xi $ sull'ipersuperficie definita da $ w = 0 $ assume la forma:
\[
  \xi = \frac{1}{4m} v \pfrac{v}
\]
Mentre il vettore normale $ n $:
\[
  n = h \pfrac{v}
\]
Questi sono paralleli, quindi $ \xi $ è un vettore normale a una ipersuperficie nulla, che quindi è un'orizzonte
di Killing. Cerco $ h $ in modo che $ n $ sia il vettore tangente ad una geodetica parametrizzata affinemente.
Si dimostra che tale $ h $ deve soddisfare:
\[
  n^\mu n^\alpha \pd_\alpha \ln h + \frac{1}{2}\pd^\mu n^2 = 0
\]
Infatti usando la definizione di vettore normale $ n^\mu = h g^{\mu\nu} \nabla_\nu K $:
\begin{align*}
  n^\alpha \nabla_\alpha n^\mu & = g^{\mu\nu} \pd_\nu n^\alpha \pd_\alpha K + g^{\mu\nu} h n^\rho \nabla_\rho \pd_\nu K = \\
                               & = n^\mu n^\alpha \pd_\alpha \ln h + g^{\mu\nu} h n^\rho \nabla_\nu \nabla_\rho K = \\
                               & = n^\mu n^\alpha \pd_\alpha \ln h - \cancel{n^2 \pd^\mu \ln h} + \frac{1}{2}\pd^\mu n^2
\end{align*}
La geodetica è parametrizzata affinemente se:
\[
  n^\alpha \nabla_\alpha n^\mu = 0 \Leftrightarrow
  n^\mu n^\alpha \pd_\alpha \ln h + \frac{1}{2}\pd^\mu n^2 = 0
\]
In questo caso scegliendo $ h = \cost $ il primo termine è sempre nullo mentre il secondo termine è ovunque proporzionale a $ w $,
e quindi valutato sull'ipersuperficie si annulla. Scelgo $ h = \frac{1}{4m} $.
% La costante di proporzionalità tra $ n $ e $ \xi $ è la gravità superficiale $ \kappa = \frac{1}{4m} $.
La gravità superficiale è definita a partire dalla funzione di proporzionalità $ \zeta $ tra il vettore di Killing e il
quello normale $ n $ all'orizzonte tale che sia tangente ad un generatore, cioè ad una geodetica nulla parametrizzata affinemente,
cioè la funzione $ \zeta $ tale che $ \xi = \zeta n $. Generalmente si utilizza la convenzione di scegliere la normalizzazione
di $ \xi $ in modo che all'infinito $ \xi^\mu\xi_\nu - 1 $, ma in questo caso il modulo del vettore di Killing $ \xi $ vale:
\begin{align*}
  \xi^2 & = \xi^\mu\xi^\nu g_{\mu\nu} = (\xi^v)^2g_{vv} + 2\xi^v\xi^wg_{vw} = \frac{v^2}{16 m^2}\frac{8mw^2}{2m + vw} - \frac{8m vw}{16m^2} = \\
        & = \frac{vw}{2m} \left( \frac{vw}{2m + vw} - 1 \right) = vw \left( \frac{vw - 2m - vw}{2m + vw}  \right) = - \frac{vw}{2m + vw} = \\
        & = - \frac{r - 2m}{r} \overset{r \to + \infty}{\rightarrow} -1
\end{align*}
Quindi il modulo si annulla su $ r = 2m $, che equivale a $ w = 0 $ o $ v = 0 $ e il vettore di Killing è
correttamente normalizzato per poter calcolare la gravità superficiale.
Per questo la gravità superficiale $ \kappa $ di questo orizzonte vale:
\[
  \kappa = \xi^\nu \pd_\nu \ln \zeta \rvert_\mathcal{N} = \frac{v}{4m}\partial_v \ln v = \frac{1}{4m}
\] \\
Per quanto riguarda le ipersuperfici definite $ v = \cost $, se il vettore normale è $ s $,
utilizzando ancora l'espressione \eqref{eq:vettore_normale} l'unico contributo proviene da $ \nu = v $, per cui:
\[
  s^\mu(v,w,\theta,\varphi) = f(v,w,\theta,\varphi) g^{\mu v} %\pfrac{w}
  \quad  \rightarrow \quad
  s(v,w,\theta,\varphi) = f(v,w,\theta,\varphi) \frac{1}{4m} \pfrac{w}
\]
Rinominando $ f $:
\[
  s(v,w,\theta,\varphi) = f(v,w,\theta,\varphi) \pfrac{w}
\]
Il modulo di questo vettore è:
\[
  s^{\mu}s_{\mu} = s^{\mu}s^{\nu}g_{\mu\nu} = s^w s^w g_{ww} = 0
\]
Per questo motivo tutte le ipersuperfici definite da $ v = \cost $ sono nulle. Il vettore di Killing $ \xi $ in $ v = 0 $
vale $ \xi = -\frac{1}{4m} w \pfrac{w} $, ed è quindi proporzionale a $ s $.

Come per il caso precedente, cerco $ f $ in modo che $ s $ sia il vettore tangente ad una geodetica nulla parametrizzata affinemente, tale
funzione è quella che soddisfa:
\[
  s^\mu s^\alpha \pd_\alpha \ln f + \frac{1}{2}\pd^\mu s^2 = 0
\]
Ma $ s^2 = 0 $ ovunque quindi questa equazione è soddisfatta da ogni $ f $ costante. Fisso $ f $ in modo che $ s = \frac{1}{4m}\pd_w $, in questo modo
la funzione di proporzionalità tra $ s $ e $ \xi $ è esattamente $ \zeta = w $.
La gravità superficiale vale perciò:
\[
  \kappa = \xi^\nu \pd_\nu \ln \zeta \rvert_\mathcal{N'} = -\frac{w}{4m}\pd_w \ln w = -\frac{1}{4m}
\] \\ \\
L'orizzonte di Killing per il campo $ \xi $ è quindi l'orizzonte biforcato ottenuto con $ \mathcal{N} \cup \mathcal{N}' $, e
la loro intersezione è una 2-sfera in $ v = w = 0 $, dove $ \xi = 0 $ ed è un punto fisso per le trasformazioni $ \phi $.
Si nota che è verificata le legge zero della dinamica dei buchi neri in quanto $ \kappa^2 = 1 \slash 16 m^2 $ è costante su tutto l'orizzonte. \\ \\
Sulle sfere a $ v,w $ costanti:
\[
  \mathcal{A} = \int \sqrt{g_{\theta\theta} g_{\varphi\varphi}} \d \theta \d \varphi = \int (2m + vw)^2 \sin \theta \d \theta  \d \varphi
  = 4 \pi (2m + vw)^2
\]
Definendo la coordinata radiale $ r = \sqrt{\frac{\mathcal{A}(v,w)}{4\pi}} = (2m + vw) $.
\[
  r = 2m + vw \quad \rightarrow \quad w = \frac{r-2m}{v} \quad \rightarrow \quad \d w = \frac{\d r }{v} - \frac{r - 2m}{v^2}\d v
\]
Con cui:
\[
  \d v \, \d w = \frac{\d v \d r }{v} - \frac{r - 2m}{v^2}\d v ^2
\]
Così facendo la metrica assume la forma:
\begin{align*}
  \d s^2 & = \frac{8m}{r} \left( \frac{r-2m}{v} \right)^2 \d v ^2 + 8m \left( \frac{\d v \d r }{v} - \frac{r-2m}{v^2}\d v ^2 \right) + r^2 \d \Omega^2 = \\
       & = \frac{8m(r-2m)}{v^2}\left[ \frac{r-2m}{r} - 1 \right]\d v ^2 + \frac{8m}{v}\d v \d r  +  r^2 \d \Omega^2 = \\
       & = -\frac{16m^2(r-2m)}{r v^2}\d v ^2 + \frac{8m}{v}\d v \d r  +  r^2 \d \Omega^2 = \\
       & = - 16m^2 \left( 1 - \frac{2m}{r} \right) \frac{\d v ^2}{v^2} + 8m \frac{\d v }{v}\d r  +  r^2 \d \Omega^2
\end{align*}
Possibili cambi di variabili per eliminare la dipendenza di $ v $ sono:
\[
  4m\frac{\d v }{v} = \pm du \quad \rightarrow \quad v(u) \propto \me^{\pm \frac{u}{4m}}
\]
In questo modo l'elemento di linea diventa:
\begin{equation}
  \label{eq:eddfink}
  \d s^2 = - \left( 1 - \frac{2m}{r} \right) du^2 \pm 2\d u\d r  +  r^2 \d \Omega^2
\end{equation}
Questa è la metrica di Schwarzschild nelle coordinate di Eddington-Finkelstein anticipate o ritardate a seconda della scelta del segno.
In effetti numerosi erano gli indizi che portavano a sospettare che la metrica \eqref{eq:metrica_problema_2} fosse parente della soluzione di Schwarzschild,
tra questi vi sono: la simmetria sferica, il ruolo speciale del punto $ r = 2m $ e i valori della gravità superficiale sugli orizzonti. In particolare tuttavia si
osserva che per $ r > 2m $ il vettore di Killing ha modulo negativo, e la metrica è invariante rispetto alle trasformazioni $ \phi $, che posso essere interpretatate
come traslazioni temporali, in quanto generate da un vettore di tipo tempo,  quindi per il teorema di Birkhoff ci si aspetta una soluzione di Schwarzschild.
Ponendo $ \d u = \d t \pm \frac{\d r }{1 - \frac{2m}{r}} $ si ritorna alla consueta forma:
\[
  \d s^2 = - \left( 1 - \frac{2m}{r} \right) \d t^2 + \frac{\d r ^2}{1 - \frac{2m}{r}} + r^2 \d \Omega^2
\]
\\
Complessivamente il cambio di coordinate è:
\[
  \begin{cases}
    r = 2m + vw \\
    u = \pm 4m \ln v
  \end{cases}
\]
Sotto questo cambio di coordinate le i vettori $ \pd_v $ e $ \pd_w $ trasformano in questo modo:
\begin{align*}
  \pfrac{v} & = \frac{\pd r}{\pd v}\pfrac{v} +  \frac{\pd u}{\pd v}\pfrac{u} = w \pfrac{r} \pm \frac{4m}{v} \pfrac{u} \\
  \pfrac{w} & = \frac{\pd r}{\pd v}\pfrac{w} +  \frac{\pd u}{\pd w}\pfrac{u} = v\pfrac{r}
\end{align*}
Per cui il vettore di Killing $ \xi $:
\[
  \xi = \frac{1}{4 m} \left( v \pfrac{v} - w \pfrac{w} \right) = \frac{1}{4m} \left( vw \pfrac{r} \pm 4m \pfrac{u} - vw \pfrac{r} \right) = \pm \pfrac{u}
\]
Siccome la metrica \eqref{eq:eddfink} non dipende esplicitamente da $ u $, questo è effettivamente un vettore di Killing.
\\
Una rappresentazione grafica di questa metrica è riportata nella figura \ref{fig:krusakl}, dove ogni punto è una 2-sfera di raggio $ (2m + vw) $.
Le singolarità di curvatura sono indicate con linee ondulate e sono definite
dall'equazione $ vw = 2m $, mentre le iperboli a $ r = \cost $, definite dall'equazione $ vw = r - 2m $, sono riportate colorate, in rosso
se $ r > 2m $, in blu in caso contrario. La sfera di biforcazione è formata dai punti fissi sotto l'azione del campo di Killing, quindi
dalla 2-sfera $ \mathcal{B} $ di raggio $ 2m $ in $ v = w = 0 $. L'orizzonte di Killing è formato dagli assi aventi per equazioni $ w = 0 $
e $ v = 0 $, come dimostrato nelle pagine precedenti.
Si nota che l'azione dell'isometria generata dal vettore di Killing mantiene inalterato il prodotto $ vw $, quindi il campo di Killing agisce mandando
le iperboli a $ r = \cost $ in sé stesse, dato che $ r = vw - 2m $. Per tracciare le orbite di $ \xi $ conviene partire osservando l'azione del campo di Killing
sulle curve con $ w = 0 $ o con $ v = 0 $ in quanto su queste si riduce ad una dilatazione,
ad esempio su $ w = 0 $ o su $ v = 0 $ l'isometria agisce come:
\[
  (v,0) \mapsto (K_\lambda v, 0) \qquad  (0,w) \mapsto (0, \frac{w}{K_\lambda})
\]
con $ K_\lambda $ costante. Assumendo $ K_\lambda > 1 \; (\Leftrightarrow \lambda > 0) $ si ottiene l'azione del campo sugli assi: per continuità si ottiene
l'azione sulle curva con $ r = \cost $.
Come già evidenziato le superfici a $ v = \cost $ sono tutte nulle, mentre quelle a $ w = \cost \not = 0 $ sono di tipo spazio.
Questo porta ad una differenza sostanziale rispetto al diagramma di Kruskal, in quanto la figura \ref{fig:krusakl} non ha la struttura causale del Minkowski,
mentre il diagramma di Kruskal è conformemente piatto.

\pgfkeys{/pgfplots/axis labels at tip/.style={
    xlabel style={at={(current axis.right of origin)}, xshift=1.5ex, anchor=center},
    ylabel style={at={(current axis.above origin)}, yshift=1.5ex, anchor=center}}
}
\pgfkeys{/pgfplots/scale/.style={
  x post scale=#1,
  y post scale=#1,
  z post scale=#1}
}

\begin{figure}[htbp]
  \centering
  % \begin{subfigure}[b]{1\linewidth}
    % \centering
    \begin{tikzpicture}
      \begin{axis}
        [
        % xmax=5,
        % ymax=5,
        samples =50,
        xlabel = $ w $,
        ylabel = $ v $,
        ticks = none,
        axis lines = center,
        axis labels at tip,
        scale=1.55,
        clip=false,
        ]
        \addplot[domain = 0.4:4.8, name path=right, thick, snake it] (x, -2/x);
        \path[name path=down] (axis cs:0.4,-5) -- (axis cs:5,-5);
        \addplot[domain = -4.8:-0.4, name path=left, thick, snake it] (x, -2/x);
        \path[name path=up] (axis cs:-4.8,5) -- (axis cs:-0.4,5);

        % \addplot[domain = 0.3:4.8, blue!100, ] (x, -1.5/x);
        \addplot[domain = 0.2:4.8, blue!80, ] (x, -1/x);
        \addplot[domain = 0.1:4.8, blue!40, ] (x, -0.5/x);
        \addplot[domain = 0.05:4.8, blue!20, ] (x, -0.25/x);
        \addplot[domain = 0.025:4.8, blue!10, ] (x, -0.125/x);

        % \addplot[domain = -0.3:-4.8, blue!100, ] (x, -1.5/x);
        \addplot[domain = -0.2:-4.8, blue!80, ] (x, -1/x);
        \addplot[domain = -0.1:-4.8, blue!40, ] (x, -0.5/x);
        \addplot[domain = -0.05:-4.8, blue!20, ] (x, -0.25/x);
        \addplot[domain = -0.025:-4.8, blue!10, ] (x, -0.125/x);

        \addplot[domain = 0.3:4.8, red!100, ] (x, 1.5/x);
        \addplot[domain = 0.2:4.8, red!80, ] (x, 1/x);
        \addplot[domain = 0.1:4.8, red!40, ] (x, 0.5/x);
        \addplot[domain = 0.05:4.8, red!20, ] (x, 0.25/x);
        \addplot[domain = 0.025:4.8, red!10, ] (x, 0.125/x);

        \addplot[domain = -0.3:-4.8, red!100, ] (x, 1.5/x);
        \addplot[domain = -0.2:-4.8, red!80, ] (x, 1/x);
        \addplot[domain = -0.1:-4.8, red!40, ] (x, 0.5/x);
        \addplot[domain = -0.05:-4.8, red!20, ] (x, 0.25/x);
        \addplot[domain = -0.025:-4.8, red!10, ] (x, 0.125/x);

        \draw[-<] (axis cs: 0,0) -- (axis cs: 0.5,0);
        \draw[-<] (axis cs: 0,0) -- (axis cs: 1,0);
        \draw[-<] (axis cs: 0,0) -- (axis cs: 1.5,0);
        \draw[-<] (axis cs: 0,0) -- (axis cs: 2,0);
        \draw[-<] (axis cs: 0,0) -- (axis cs: 2.5,0);
        \draw[-<] (axis cs: 0,0) -- (axis cs: 3,0);
        \draw[-<] (axis cs: 0,0) -- (axis cs: 3.5,0);
        \draw[-<] (axis cs: 0,0) -- (axis cs: 4,0);
        \draw[-<] (axis cs: 0,0) -- (axis cs: 4.5,0);

        \draw[-<] (axis cs: 0,0) -- (axis cs: -0.5,0);
        \draw[-<] (axis cs: 0,0) -- (axis cs: -1,0);
        \draw[-<] (axis cs: 0,0) -- (axis cs: -1.5,0);
        \draw[-<] (axis cs: 0,0) -- (axis cs: -2,0);
        \draw[-<] (axis cs: 0,0) -- (axis cs: -2.5,0);
        \draw[-<] (axis cs: 0,0) -- (axis cs: -3,0);
        \draw[-<] (axis cs: 0,0) -- (axis cs: -3.5,0);
        \draw[-<] (axis cs: 0,0) -- (axis cs: -4,0);
        \draw[-<] (axis cs: 0,0) -- (axis cs: -4.5,0);

        \draw[->] (axis cs: 0,0) -- (axis cs: 0,0.5);
        \draw[->] (axis cs: 0,0) -- (axis cs: 0,1);
        \draw[->] (axis cs: 0,0) -- (axis cs: 0,1.5);
        \draw[->] (axis cs: 0,0) -- (axis cs: 0,2);
        \draw[->] (axis cs: 0,0) -- (axis cs: 0,2.5);
        \draw[->] (axis cs: 0,0) -- (axis cs: 0,3);
        \draw[->] (axis cs: 0,0) -- (axis cs: 0,3.5);
        \draw[->] (axis cs: 0,0) -- (axis cs: 0,4);
        \draw[->] (axis cs: 0,0) -- (axis cs: 0,4.5);

        \draw[->] (axis cs: 0,0) -- (axis cs: 0,-0.5);
        \draw[->] (axis cs: 0,0) -- (axis cs: 0,-1);
        \draw[->] (axis cs: 0,0) -- (axis cs: 0,-1.5);
        \draw[->] (axis cs: 0,0) -- (axis cs: 0,-2);
        \draw[->] (axis cs: 0,0) -- (axis cs: 0,-2.5);
        \draw[->] (axis cs: 0,0) -- (axis cs: 0,-3);
        \draw[->] (axis cs: 0,0) -- (axis cs: 0,-3.5);
        \draw[->] (axis cs: 0,0) -- (axis cs: 0,-4);
        \draw[->] (axis cs: 0,0) -- (axis cs: 0,-4.5);

        \draw[-<] (axis cs:  0.3 ,  3.22 ) -- (axis cs:  0.32 ,  3.12 );
        \draw[-<] (axis cs:  0.5 ,  2.07 ) -- (axis cs:  0.53 ,  1.95 );
        \draw[-<] (axis cs:  0.72 ,  1.37 ) -- (axis cs:  0.8 ,  1.25 );
        \draw[-<] (axis cs:  1.16 ,  0.86 ) -- (axis cs:  1.28 ,  0.78 );
        \draw[-<] (axis cs:  1.64 ,  0.61 ) -- (axis cs:  1.76 ,  0.57 );
        \draw[-<] (axis cs:  2.12 ,  0.47 ) -- (axis cs:  2.24 ,  0.45 );
        \draw[-<] (axis cs:  2.6 ,  0.38 ) -- (axis cs:  2.72 ,  0.37 );
        \draw[-<] (axis cs:  3.08 ,  0.32 ) -- (axis cs:  3.2 ,  0.31 );
        \draw[-<] (axis cs:  3.56 ,  0.28 ) -- (axis cs:  3.68 ,  0.27 );
        \draw[-<] (axis cs:  4.04 ,  0.25 ) -- (axis cs:  4.16 ,  0.24 );
        \draw[-<] (axis cs:  4.52 ,  0.22 ) -- (axis cs:  4.64 ,  0.22 );

        \draw[-<] (axis cs:  0.3 ,  -3.22 ) -- (axis cs:  0.32 ,  -3.12 );
        \draw[-<] (axis cs:  0.5 ,  -2.07 ) -- (axis cs:  0.53 ,  -1.95 );
        \draw[-<] (axis cs:  0.72 ,  -1.37 ) -- (axis cs:  0.8 ,  -1.25 );
        \draw[-<] (axis cs:  1.16 ,  -0.86 ) -- (axis cs:  1.28 ,  -0.78 );
        \draw[-<] (axis cs:  1.64 ,  -0.61 ) -- (axis cs:  1.76 ,  -0.57 );
        \draw[-<] (axis cs:  2.12 ,  -0.47 ) -- (axis cs:  2.24 ,  -0.45 );
        \draw[-<] (axis cs:  2.6 ,  -0.38 ) -- (axis cs:  2.72 ,  -0.37 );
        \draw[-<] (axis cs:  3.08 ,  -0.32 ) -- (axis cs:  3.2 ,  -0.31 );
        \draw[-<] (axis cs:  3.56 ,  -0.28 ) -- (axis cs:  3.68 ,  -0.27 );
        \draw[-<] (axis cs:  4.04 ,  -0.25 ) -- (axis cs:  4.16 ,  -0.24 );
        \draw[-<] (axis cs:  4.52 ,  -0.22 ) -- (axis cs:  4.64 ,  -0.22 );

        \draw[-<] (axis cs:  -0.3 ,  -3.22 ) -- (axis cs:  -0.32 ,  -3.12 );
        \draw[-<] (axis cs:  -0.5 ,  -2.07 ) -- (axis cs:  -0.53 ,  -1.95 );
        \draw[-<] (axis cs:  -0.72 ,  -1.37 ) -- (axis cs:  -0.8 ,  -1.25 );
        \draw[-<] (axis cs:  -1.16 ,  -0.86 ) -- (axis cs:  -1.28 ,  -0.78 );
        \draw[-<] (axis cs:  -1.64 ,  -0.61 ) -- (axis cs:  -1.76 ,  -0.57 );
        \draw[-<] (axis cs:  -2.12 ,  -0.47 ) -- (axis cs:  -2.24 ,  -0.45 );
        \draw[-<] (axis cs:  -2.6 ,  -0.38 ) -- (axis cs:  -2.72 ,  -0.37 );
        \draw[-<] (axis cs:  -3.08 ,  -0.32 ) -- (axis cs:  -3.2 ,  -0.31 );
        \draw[-<] (axis cs:  -3.56 ,  -0.28 ) -- (axis cs:  -3.68 ,  -0.27 );
        \draw[-<] (axis cs:  -4.04 ,  -0.25 ) -- (axis cs:  -4.16 ,  -0.24 );
        \draw[-<] (axis cs:  -4.52 ,  -0.22 ) -- (axis cs:  -4.64 ,  -0.22 );

        \draw[-<] (axis cs:  -0.3 ,  3.22 ) -- (axis cs:  -0.32 ,  3.12 );
        \draw[-<] (axis cs:  -0.5 ,  2.07 ) -- (axis cs:  -0.53 ,  1.95 );
        \draw[-<] (axis cs:  -0.72 ,  1.37 ) -- (axis cs:  -0.8 ,  1.25 );
        \draw[-<] (axis cs:  -1.16 ,  0.86 ) -- (axis cs:  -1.28 ,  0.78 );
        \draw[-<] (axis cs:  -1.64 ,  0.61 ) -- (axis cs:  -1.76 ,  0.57 );
        \draw[-<] (axis cs:  -2.12 ,  0.47 ) -- (axis cs:  -2.24 ,  0.45 );
        \draw[-<] (axis cs:  -2.6 ,  0.38 ) -- (axis cs:  -2.72 ,  0.37 );
        \draw[-<] (axis cs:  -3.08 ,  0.32 ) -- (axis cs:  -3.2 ,  0.31 );
        \draw[-<] (axis cs:  -3.56 ,  0.28 ) -- (axis cs:  -3.68 ,  0.27 );
        \draw[-<] (axis cs:  -4.04 ,  0.25 ) -- (axis cs:  -4.16 ,  0.24 );
        \draw[-<] (axis cs:  -4.52 ,  0.22 ) -- (axis cs:  -4.64 ,  0.22 );

        \addplot[thick] (x, 0);
        \addplot[thick] (0, x);

        % \addplot[shading = axis, left color = black!30, right color = black!01,
        % shading angle = 45] fill between[of = right and down, soft clip={domain=0.4:4.8}];
        % \addplot[shading = axis, left color = black!01, right color = black!30,
        % shading angle = 45] fill between[of = left and up]; %soft clip={domain=-4.8:-0.4}];
        % \addplot[shading = axis, left color = black!70, right color = black!05,
        % shading angle = 45]
        % fill between[of = right and down, soft clip={domain=0.4:4.8}];

        \node[right] at (axis cs:4.8,-0.4) { $ r = 0 $ };
        \node[left] at (axis cs:-4.8,0.4) { $ r = 0 $ };
        \node[left] at (axis cs:-5,0.) { $ r = 2m $ };
        \node[below] at (axis cs:0,-5) { $ r = 2m $ };
        \node at (axis cs:3,-3) { $ r < 0 $ };
        \node at (axis cs:-3,3) { $ r < 0 $ };
        \node[above right]  at (axis cs:0,0) { $ \mathcal{B} $ };
        \fill (axis cs:0,0) circle [radius=2pt];

      \end{axis}
    \end{tikzpicture}
    \caption{Diagramma di Kruskal per la metrica \eqref{eq:metrica_problema_2}. Le iperboli rosse sono curve a raggio costante con $ r > 2m $,
      quelle blu con $ r < 2m $. $ \mathcal{B} $ è il punto di biforcazione dell'orizzonte di Killing del campo $ \xi $. Le isometrie $ \phi_\lambda $
      consistono in rotazioni iperboliche lungo le curve a $ r = \cost $. In questa rappresentazione i coni luce \emph{non} sono gli stessi del Minkowski.}
    \label{fig:krusakl}
    % \vspace*{-50pt}
  % \end{subfigure}
  % \begin{subfigure}[b]{1\linewidth}
  %   \centering
  %   \begin{tikzpicture} [rotate = -45]
  %     \begin{axis}
  %       [
  %       xmin=-3,
  %       ymin=-3,
  %       xmax=3,
  %       ymax=3,
  %       samples =50,
  %       % xlabel = $ w $,
  %       % ylabel = $ v $,
  %       ticks = none,
  %       unit vector ratio*=1 1 1,
  %       axis lines* = center,
  %       axis labels at tip,
  %       scale=1.5,
  %       % clip=false,
  %       ]
  %       \addplot[domain = 0.1:4, name path=right, thick, snake it] (x, -1/x);
  %       % \path[name path=down] (axis cs:0.4,-4.8) -- (axis cs:4.8,-4.8);
  %       \addplot[domain = -4:-0.1, name path=left, thick, snake it] (x, -1/x);
  %       % \path[name path=up] (axis cs:-4.8,4.8) -- (axis cs:-0.4,4.8);

  %       % \addplot[domain = 0.3:4.8, blue!100, ] (x, -1.5/x);
  %       % \addplot[domain = 0.1:4.8, blue!80, ] (x, -1/x);
  %       % \addplot[domain = 0.1:4.8, blue!40, ] (x, -0.5/x);
  %       % \addplot[domain = 0.05:4.8, blue!20, ] (x, -0.25/x);
  %       % \addplot[domain = 0.025:4.8, blue!10, ] (x, -0.125/x);

  %       % \addplot[domain = -0.3:-4.8, blue!100, ] (x, -1.5/x);
  %       % \addplot[domain = -0.1:-4.8, blue!80, ] (x, -1/x);
  %       % \addplot[domain = -0.1:-4.8, blue!40, ] (x, -0.5/x);
  %       % \addplot[domain = -0.05:-4.8, blue!20, ] (x, -0.25/x);
  %       % \addplot[domain = -0.025:-4.8, blue!10, ] (x, -0.125/x);

  %       % \addplot[domain = 0.3:4.8, red!100, ] (x, 1.5/x);
  %       \addplot[domain = 0.2:4, red!80, thick] (x, 1/x);
  %       % \addplot[domain = 0.1:4.8, red!40, ] (x, 0.5/x);
  %       % \addplot[domain = 0.05:4.8, red!20, ] (x, 0.25/x);
  %       % \addplot[domain = 0.025:4.8, red!10, ] (x, 0.125/x);

  %       % \addplot[domain = -0.3:-4.8, red!100, ] (x, 1.5/x);
  %       \addplot[domain = -0.2:-4, red!80, thick] (x, 1/x);
  %       % \addplot[domain = -0.1:-4.8, red!40, ] (x, 0.5/x);
  %       % \addplot[domain = -0.05:-4.8, red!20, ] (x, 0.25/x);
  %       % \addplot[domain = -0.025:-4.8, red!10, ] (x, 0.125/x);

  %       % \draw[->] (axis cs: 0,0) -- (axis cs: 0.5,0);
  %       % \draw[->] (axis cs: 0,0) -- (axis cs: 1,0);
  %       % \draw[->] (axis cs: 0,0) -- (axis cs: 1.5,0);
  %       % \draw[->] (axis cs: 0,0) -- (axis cs: 2,0);
  %       % \draw[->] (axis cs: 0,0) -- (axis cs: 2.5,0);
  %       % \draw[->] (axis cs: 0,0) -- (axis cs: 3,0);
  %       % \draw[->] (axis cs: 0,0) -- (axis cs: 3.5,0);
  %       % \draw[->] (axis cs: 0,0) -- (axis cs: 4,0);
  %       % \draw[->] (axis cs: 0,0) -- (axis cs: 4.5,0);

  %       % \draw[->] (axis cs: 0,0) -- (axis cs: -0.5,0);
  %       % \draw[->] (axis cs: 0,0) -- (axis cs: -1,0);
  %       % \draw[->] (axis cs: 0,0) -- (axis cs: -1.5,0);
  %       % \draw[->] (axis cs: 0,0) -- (axis cs: -2,0);
  %       % \draw[->] (axis cs: 0,0) -- (axis cs: -2.5,0);
  %       % \draw[->] (axis cs: 0,0) -- (axis cs: -3,0);
  %       % \draw[->] (axis cs: 0,0) -- (axis cs: -3.5,0);
  %       % \draw[->] (axis cs: 0,0) -- (axis cs: -4,0);
  %       % \draw[->] (axis cs: 0,0) -- (axis cs: -4.5,0);

  %       % \draw[-<] (axis cs: 0,0) -- (axis cs: 0,0.5);
  %       % \draw[-<] (axis cs: 0,0) -- (axis cs: 0,1);
  %       % \draw[-<] (axis cs: 0,0) -- (axis cs: 0,1.5);
  %       % \draw[-<] (axis cs: 0,0) -- (axis cs: 0,2);
  %       % \draw[-<] (axis cs: 0,0) -- (axis cs: 0,2.5);
  %       % \draw[-<] (axis cs: 0,0) -- (axis cs: 0,3);
  %       % \draw[-<] (axis cs: 0,0) -- (axis cs: 0,3.5);
  %       % \draw[-<] (axis cs: 0,0) -- (axis cs: 0,4);
  %       % \draw[-<] (axis cs: 0,0) -- (axis cs: 0,4.5);

  %       % \draw[-<] (axis cs: 0,0) -- (axis cs: 0,-0.5);
  %       % \draw[-<] (axis cs: 0,0) -- (axis cs: 0,-1);
  %       % \draw[-<] (axis cs: 0,0) -- (axis cs: 0,-1.5);
  %       % \draw[-<] (axis cs: 0,0) -- (axis cs: 0,-2);
  %       % \draw[-<] (axis cs: 0,0) -- (axis cs: 0,-2.5);
  %       % \draw[-<] (axis cs: 0,0) -- (axis cs: 0,-3);
  %       % \draw[-<] (axis cs: 0,0) -- (axis cs: 0,-3.5);
  %       % \draw[-<] (axis cs: 0,0) -- (axis cs: 0,-4);
  %       % \draw[-<] (axis cs: 0,0) -- (axis cs: 0,-4.5);

  %       % \draw[->] (axis cs:  0.3 ,  3.22 ) -- (axis cs:  0.32 ,  3.12 );
  %       % \draw[->] (axis cs:  0.5 ,  2.07 ) -- (axis cs:  0.53 ,  1.95 );
  %       % \draw[->] (axis cs:  0.72 ,  1.37 ) -- (axis cs:  0.8 ,  1.25 );
  %       % \draw[->] (axis cs:  1.16 ,  0.86 ) -- (axis cs:  1.28 ,  0.78 );
  %       % \draw[->] (axis cs:  1.64 ,  0.61 ) -- (axis cs:  1.76 ,  0.57 );
  %       % \draw[->] (axis cs:  2.12 ,  0.47 ) -- (axis cs:  2.24 ,  0.45 );
  %       % \draw[->] (axis cs:  2.6 ,  0.38 ) -- (axis cs:  2.72 ,  0.37 );
  %       % \draw[->] (axis cs:  3.08 ,  0.32 ) -- (axis cs:  3.2 ,  0.31 );
  %       % \draw[->] (axis cs:  3.56 ,  0.28 ) -- (axis cs:  3.68 ,  0.27 );
  %       % \draw[->] (axis cs:  4.04 ,  0.25 ) -- (axis cs:  4.16 ,  0.24 );
  %       % \draw[->] (axis cs:  4.52 ,  0.22 ) -- (axis cs:  4.64 ,  0.22 );

  %       % \draw[->] (axis cs:  0.3 ,  -3.22 ) -- (axis cs:  0.32 ,  -3.12 );
  %       % \draw[->] (axis cs:  0.5 ,  -2.07 ) -- (axis cs:  0.53 ,  -1.95 );
  %       % \draw[->] (axis cs:  0.72 ,  -1.37 ) -- (axis cs:  0.8 ,  -1.25 );
  %       % \draw[->] (axis cs:  1.16 ,  -0.86 ) -- (axis cs:  1.28 ,  -0.78 );
  %       % \draw[->] (axis cs:  1.64 ,  -0.61 ) -- (axis cs:  1.76 ,  -0.57 );
  %       % \draw[->] (axis cs:  2.12 ,  -0.47 ) -- (axis cs:  2.24 ,  -0.45 );
  %       % \draw[->] (axis cs:  2.6 ,  -0.38 ) -- (axis cs:  2.72 ,  -0.37 );
  %       % \draw[->] (axis cs:  3.08 ,  -0.32 ) -- (axis cs:  3.2 ,  -0.31 );
  %       % \draw[->] (axis cs:  3.56 ,  -0.28 ) -- (axis cs:  3.68 ,  -0.27 );
  %       % \draw[->] (axis cs:  4.04 ,  -0.25 ) -- (axis cs:  4.16 ,  -0.24 );
  %       % \draw[->] (axis cs:  4.52 ,  -0.22 ) -- (axis cs:  4.64 ,  -0.22 );

  %       % \draw[->] (axis cs:  -0.3 ,  -3.22 ) -- (axis cs:  -0.32 ,  -3.12 );
  %       % \draw[->] (axis cs:  -0.5 ,  -2.07 ) -- (axis cs:  -0.53 ,  -1.95 );
  %       % \draw[->] (axis cs:  -0.72 ,  -1.37 ) -- (axis cs:  -0.8 ,  -1.25 );
  %       % \draw[->] (axis cs:  -1.16 ,  -0.86 ) -- (axis cs:  -1.28 ,  -0.78 );
  %       % \draw[->] (axis cs:  -1.64 ,  -0.61 ) -- (axis cs:  -1.76 ,  -0.57 );
  %       % \draw[->] (axis cs:  -2.12 ,  -0.47 ) -- (axis cs:  -2.24 ,  -0.45 );
  %       % \draw[->] (axis cs:  -2.6 ,  -0.38 ) -- (axis cs:  -2.72 ,  -0.37 );
  %       % \draw[->] (axis cs:  -3.08 ,  -0.32 ) -- (axis cs:  -3.2 ,  -0.31 );
  %       % \draw[->] (axis cs:  -3.56 ,  -0.28 ) -- (axis cs:  -3.68 ,  -0.27 );
  %       % \draw[->] (axis cs:  -4.04 ,  -0.25 ) -- (axis cs:  -4.16 ,  -0.24 );
  %       % \draw[->] (axis cs:  -4.52 ,  -0.22 ) -- (axis cs:  -4.64 ,  -0.22 );

  %       % \draw[->] (axis cs:  -0.3 ,  3.22 ) -- (axis cs:  -0.32 ,  3.12 );
  %       % \draw[->] (axis cs:  -0.5 ,  2.07 ) -- (axis cs:  -0.53 ,  1.95 );
  %       % \draw[->] (axis cs:  -0.72 ,  1.37 ) -- (axis cs:  -0.8 ,  1.25 );
  %       % \draw[->] (axis cs:  -1.16 ,  0.86 ) -- (axis cs:  -1.28 ,  0.78 );
  %       % \draw[->] (axis cs:  -1.64 ,  0.61 ) -- (axis cs:  -1.76 ,  0.57 );
  %       % \draw[->] (axis cs:  -2.12 ,  0.47 ) -- (axis cs:  -2.24 ,  0.45 );
  %       % \draw[->] (axis cs:  -2.6 ,  0.38 ) -- (axis cs:  -2.72 ,  0.37 );
  %       % \draw[->] (axis cs:  -3.08 ,  0.32 ) -- (axis cs:  -3.2 ,  0.31 );
  %       % \draw[->] (axis cs:  -3.56 ,  0.28 ) -- (axis cs:  -3.68 ,  0.27 );
  %       % \draw[->] (axis cs:  -4.04 ,  0.25 ) -- (axis cs:  -4.16 ,  0.24 );
  %       % \draw[->] (axis cs:  -4.52 ,  0.22 ) -- (axis cs:  -4.64 ,  0.22 );

  %       % \addplot[thick] (x, 0);
  %       % \addplot[thick] (0, x);

  %       % \addplot[shading = axis, left color = black!30, right color = black!01,
  %       % shading angle = 45] fill between[of = right and down, soft clip={domain=0.4:4.8}];
  %       % \addplot[shading = axis, left color = black!01, right color = black!30,
  %       % shading angle = 45] fill between[of = left and up]; %soft clip={domain=-4.8:-0.4}];
  %       % \addplot[shading = axis, left color = black!70, right color = black!05,
  %       % shading angle = 45]
  %       % fill between[of = right and down, soft clip={domain=0.4:4.8}];

  %       \node[above, rotate = 45] at (axis cs:-0.2,0.2) { II };
  %       \node[below, rotate = 45] at (axis cs:0.2,-0.2) { IV };
  %       \node[left, rotate = 45] at (axis cs:-0.2,-0.2) { III };
  %       \node[right, rotate = 45] at (axis cs:0.2,0.2) { I };

  %       \node[rotate = 150] at (axis cs:1.2,0.6) { $ r = \cost $ };
  %       \node[above] at (axis cs:-2.4,0.) { $ r = 2m $ };
  %       \node[rotate = 90, above] at (axis cs:0,-2.4) { $ r = 2m $ };
  %       \node[rotate = 45] at (axis cs:1.5,-1.5) { $ r < 0 $ };
  %       \node[rotate = 45] at (axis cs:-1.5,1.5) { $ r < 0 $ };

  %       % \fill (axis cs:0,0) circle [radius=2pt];

  %     \end{axis}
  %   \end{tikzpicture}
  %   \vspace*{-50pt}
  %   \caption{Diagramma ruotato di $ \SI{45}{\degree} $ secondo le usuali consuetudini di rappresentazione. Le regioni
  %     numerate solo le medesime regioni del Kruskal.}
  %   \label{fig:krusakl2}
  % \end{subfigure}
\end{figure}

\section*{Esercizio 3}

\subsection*{Testo}
Si consideri uno spaziotempo $ d + 1 $-dimensionale con la metrica
\begin{equation}
  \label{eq:metrica_problema_3}
  \d{\hat{s}^2_{d+1}} = \me^{2\alpha \Phi} \d {s^2_d} + \me^{2 \beta \Phi} \left( \d z + A \right)^2
\end{equation}
dove $ z \sim z + 2\pi L $ e
\[
  \alpha^2 = \frac{1}{2 \left( d - 1 \right) \left( d - 2 \right) }, \qquad \beta = - \left( d - 2 \right) \alpha
\]
Il dilatone $ \Phi $, il campo di Maxwell $ A = A_\mu dx^\mu $ e la metrica $d$-dimensionale $ \d{s^2_d} $ non dipendono dalla coordinata
$ z $. $ \mu, \nu, \dots $ sono indici curvi in $ d $ dimensioni, mentre $ M, N, \dots $ si riferiscono a $ d + 1 $ dimensioni.

Si dimostri che l'equazione di Klein-Gordon
\begin{equation}
  \label{eq:klein_gordon}
  \pd_M \left( \hg^{MN}\sqrt{-\hg} \pd_N \hat{\phi} \right) = 0
\end{equation}
dà luogo all'equazione
\begin{equation}
  \label{eq:equazione_es_3}
  \left( \nabla_\mu - \frac{in}{L} A_\mu \right) \left( \nabla^\mu - \frac{in}{L} A^\mu  \right) \phi_n = \me^{2 \left(d-1\right) \alpha \Phi}\frac{n^2}{L^2}\phi_n,
\end{equation}
dove $ \phi_n $ denota il componente di Fourier nello sviluppo:
\[
\hat{\phi}(x^\mu, z) = \sum_n \phi_n(x^\mu) \me^{\frac{inz}{L}}
\]
Quali sono la massa e la carica del campo $ \phi_n $?

\subsection*{Soluzione}
In quello che segue si utilizza la convenzione sui cappucci e sugli indici per spazi $ d $ e $ d+1 $-dimensionali: i cappucci e le lettere latine maiuscole
si riferiscono a grandezze nello spazio $ d+1 $-dimensionale, mentre le lettere greche e l'assenza di cappucci a quelle nello spazio $ d $-dimensionale.
Per prima cosa dimostro che l'equazione \eqref{eq:klein_gordon} è effettivamente l'equazione di Klein-Gordon in background curvilineo:
$ \hat{\nabla}_M \hat{\nabla}^M \hat{\phi} = 0 $.
Per far ciò utilizzo un paio di lemmi:
\begin{lemma}
  \label{lem:traccia_simboli}
  Vale che:
  \begin{equation}
    \label{eq:traccia_cristhoffel}
    \Gamma^\mu_{\;\mu\nu} = \pd_\nu \sg
  \end{equation}
\end{lemma}

\begin{proof}
  Dalla relazione che lega i simboli di Cristhoffel alla metrica:
  % \begin{align*}
  \[
    \Gamma^\mu_{\;\mu\nu}  = \frac{1}{2}g^{\mu\alpha} (\pd_\nu g_{\alpha\mu} + \pd_\mu g_{\nu\alpha} - \pd_\alpha g_{\mu\nu} ) \\
                          = \frac{1}{2} \left( g^{\mu\alpha} \pd_\nu g_{\alpha\mu} + \cancel{\pd^\alpha g_{\nu\alpha}} - \cancel{\pd^\mu g_{\mu\nu}} \right) \\
                          =  \frac{1}{2}g^{\mu\alpha}\pd_\nu g_{\mu\alpha}
                        \]
  % \end{align*}
  Nel quale ho utilizzato nel secondo passaggio che la metrica è simmetrica.
  Ma vale anche che:
  \[
    \pd_\nu \ln \sqrt{-g} = \frac{1}{\sqrt{-g}}\pd_\nu\sqrt{-g} = \frac{1}{2}g^{\mu\alpha}\pd_\nu g_{\mu\alpha}
  \]
  Da cui per confronto risulta la \eqref{eq:traccia_cristhoffel}.
\end{proof}
\noindent
Utilizzando il lemma \ref{lem:traccia_simboli} dimostro un secondo utile risultato:
\begin{lemma}
  \label{lem:divergenza_vettore}
  Sia $ A $ un vettore, allora:
  \[
    \nabla_\mu A^\mu = \frac{1}{\sg} \pd_\mu(\sg A^\mu)
  \]
\end{lemma}
\begin{proof}
  Dalla definizione di derivata covariante:
  \begin{align*}
    \nabla_\mu A^\mu  = \pd_\mu A^\mu + \Gamma^\mu_{\;\mu\nu}A^\nu
                      \overset{\eqref{eq:traccia_cristhoffel}}{=} \pd_\mu A^\mu +  \frac{1}{\sqrt{-g}}\pd_\mu\sqrt{-g}A^\mu
                      = \frac{1}{\sqrt{-g}}\pd_\mu \left( \sqrt{-g} A^\mu \right)
  \end{align*}
  Dove nell'ultimo passaggio ho riconosciuto la derivata di un prodotto.
\end{proof}
\noindent
Utilizzando questo lemma sull'equazione di Klein-Gordon in un background curvo:
\[
  \hat{\nabla}_M \hat{\nabla}^M \hat{\phi} =  \hat{\nabla}_M \pd^M \hat{\phi} =
  \frac{1}{\sqrt{-\hg}} \left( \sqrt{-\hg}\pd^N \hat{\phi}  \right) =
  \frac{1}{\sqrt{-\hg}} \left( \hg^{MN}\sqrt{-\hg}\pd_N \hat{\phi}  \right)
\]
E quindi ho riscritto nella forma dell'esercizio l'equazione di Klein-Gordon.
Manipolo quindi l'espressione:
\begin{multline*}
  \pd_M(\hg^{MN}\sqrt{-\hg}\pd_N\hat{\phi}) =
  \pd_\mu(\hg^{\mu\nu}\sqrt{-\hg}\pd_\nu\hat{\phi}) +
  \pd_\mu(\hg^{\mu z}\sqrt{-\hg}\pd_z\hat{\phi}) +
  \pd_z(\hg^{z \mu}\sqrt{-\hg}\pd_\mu\hat{\phi}) + \\
  \pd_z(\hg^{z z}\sqrt{-\hg}\pd_z\hat{\phi})
\end{multline*}
Sviluppando con l'ansatz di Kaluza-Klein:
\[
  \hg^{\mu\nu} = \me^{-2 \alpha \Phi} g^{\mu\nu} \quad
  \hg^{\mu z} = -\me^{-2 \alpha \Phi} A^\mu \quad
  \hg^{z z} = A_\mu A^\mu \me^{-2 \alpha \Phi} + \me^{-2\beta \Phi}
\]
Da cui:
\[
  \sqrt{-\hg} = \me^{\beta + d\alpha}\sg
\]
Ma $ \beta = - (d -2)\alpha $, quindi $ \sqrt{-\hg} = \me^{2\alpha\Phi}\sg $.
Si ottiene perciò:
\begin{gather*}
 = {}   \pd_\mu (\me^{-2\alpha\Phi} g^{\mu\nu} \me^{2\alpha\Phi} \sg \pd_z \hat{\phi} ) + \pd_\mu (-\me^{-2\alpha\Phi}A^\mu \me^{2\alpha\Phi} \sg \pd_z \hat{\phi} )  +  \pd_z (-\me^{-2\alpha\Phi} A^\mu \me^{2\alpha\Phi} \sg \pd_\mu \hat{\phi} )       \\
       + \pd_z \left((A_\mu A^\mu\me^{-2\alpha\Phi} + \me^{2(d-2)\alpha\phi})\me^{2\alpha\Phi} \sg \pd_z \hat{\phi} \right)  = \pd_\mu (g^{\mu\nu}\sg\pd_\nu \hat{\phi}) - \pd_\mu (A^\mu \sg \pd_z \hat{\phi})  \\
    - \pd_z (A^\mu \sg\pd_\mu \hat{\phi}) + \pd_z (A_\mu A^\mu \sg\pd_z \hat{\phi})     + \pd_z (\me^{2(d-1)\alpha\Phi}\sg\pd_z\hat{\phi})  =
\end{gather*}
Utilizzando il fatto che le grandezze con indici greci non dipendono da $ z $:
\[
  = \pd_\mu (g^{\mu\nu}\sg\pd_\nu \hat{\phi}) - \pd_\mu(A^\mu \sg \pd_z \hat{\phi})
  - A^\mu\sg \pd_z \pd_\mu \hat{\phi} + A^2 \sg \pd_z^2 \hat{\phi} +                                                                                                                                           \me^{2(d-1)\alpha\Phi} \sg \pd_z^2 \hat{\phi}
\]
Questa espressione è uguale a zero per ipotesi, dividendo quindi tutto per $ \sg $:
\[
  = \frac{1}{\sg}\pd_\mu (g^{\mu\nu}\sg\pd_\nu \hat{\phi}) -  \frac{1}{\sg} \pd_\mu(A^\mu \sg \pd_z \hat{\phi})
  - A^\mu \pd_z \pd_\mu \hat{\phi} + A^2 \pd_z^2 \hat{\phi} + \me^{2(d-1)\alpha\Phi} \pd_z^2 \hat{\phi}
\]
Facendo agire la derivata sul secondo termine:
\[
  = \frac{1}{\sg}\pd_\mu (g^{\mu\nu}\sg\pd_\nu \hat{\phi}) -  \frac{1}{\sg} \pd_\mu(A^\mu \sg) \pd_z \hat{\phi}
  - 2A^\mu \pd_z \pd_\mu \hat{\phi} + A^2 \pd_z^2 \hat{\phi} + \me^{2(d-1)\alpha\Phi} \pd_z^2 \hat{\phi}
\]
Utilizzando il lemma \ref{lem:divergenza_vettore} e scambiando l'ordine delle derivate nel terzo termine:
% La dimostrazione riportata all'inizio dell'esercizio per l'uguaglianza $ \hat{\nabla}_M \hat{\nabla}^M \phi = \frac{1}{\sg}\pd_N (g^{MN}\sg\pd_N \hat{\phi}) $
% può essere utilizzata allo stesso modo per dimostrare che
% \[
%   \frac{1}{\sg}\pd_\mu (g^{\mu\nu}\sg\pd_\nu \hat{\phi}) = \nabla_\mu \nabla^\mu \hat{\phi} \quad \text{ e } \quad
%   \frac{1}{\sg} \pd_\mu(A^\mu \sg)  = \nabla_\mu A^\mu
% \]
% Per cui:
\[
  \nabla^\mu\nabla_\mu \hat{\phi} - \nabla_\mu A^\mu \pd_z \hat{\phi} - 2 A^\mu \pd_\mu \pd_z \hat{\phi} + \left(A^2 + \me^{2(d-1)\alpha\Phi}\right)\pd_z^2 \hat{\phi} = 0
\]
Inserendo lo sviluppo di Fourier:
\begin{multline*}
  \nabla^\mu\nabla_\mu \sum_n \phi_n \me^{\frac{inz}{L}} - 2 A^\mu \pd_\mu \pd_z \sum_n \phi_n \me^{\frac{inz}{L}} -
  \nabla_\mu A^\mu \pd_z \sum_n \phi_n \me^{\frac{inz}{L}} + \\  (A^2 + \me^{2(d-1)\alpha}) \pd_z^2 \sum_n \phi_n \me^{\frac{inz}{L}} = 0
\end{multline*}
Sviluppando tendendo in considerazione che $ \nabla_\mu $ agisce solo sulle coordinate greche:
\begin{multline*}
  \sum_n \left( \nabla^\mu\nabla_\mu \phi_n  \right) \me^{\frac{inz}{L}} - 2 A^\mu \sum_n \pd_\mu \phi_n \frac{in}{L}  \me^{\frac{inz}{L}}
  - \nabla_\mu A^\mu \frac{in}{L}\sum_n \phi_n  \me^{\frac{inz}{L}} - \\ \frac{n^2}{L^2}(A^2 + \me^{2(d-1)\alpha\Phi})\sum_n \phi_n  \me^{\frac{inz}{L}} = 0
\end{multline*}
Siccome $ \set{ f_n(z) = \me^{\frac{inz}{L}}}_{n \in \mathbb{N}} $ è una base di Fourier, per $ n $ distinti queste funzioni sono linearmente indipendenti quindi la
seguente relazione deve essere vera per ogni $ n \in \mathbb{N} $:
\[
  \nabla^\mu\nabla_\mu \phi_n - 2 A^\mu \pd_\mu \phi_n \frac{in}{L}  - \nabla_\mu A^\mu \frac{in}{L} \phi_n - \frac{n^2}{L^2}(A^2 + \me^{2(d-1)\alpha\Phi}) \phi_n = 0
\]
Considerando che $ \nabla_\mu \phi = \pd_\mu \phi $ l'equazione si riscrive equivalentemente come:
\[
  \left( \nabla_\mu - \frac{in}{L} A_\mu \right) \left( \nabla^\mu - \frac{in}{L} A^\mu  \right) \phi_n = \me^{2 \left(d-1\right) \alpha \Phi}\frac{n^2}{L^2}\phi_n,
\]
Che è esattamente l'equazione \eqref{eq:equazione_es_3}. Questa è un'equazione di Klein-Gordon per un campo carico massivo.
\\ \\
In uno spaziotempo piatto si introduce l'interazione elettromagnetica rendendo locale la simmetria di gauge U(1), questo significa introdurre
una derivata covariante della forma $ D_\mu = \pd_\mu + iq A_\mu $. L'accoppiamento minimale con il campo elettromagnetico si ottiene allo stesso
modo anche in un bakground curvo, nel quale si definisce la derivata totalmente covariante (cioè covariante rispetto a diffeomorfismi e a trasformazioni
di gauge U(1) locali) $ D_\mu = \nabla_\mu + iq A_\mu $. Per confronto con $ D_\mu = \nabla_\mu - \frac{in}{L} A_\mu $ si trova che la carica del modo $ n- $esimo è
$ q = -\frac{n}{L} $.
Inoltre definendo $ \Box = D_\mu D^\mu $ l'equazione \eqref{eq:equazione_es_3} diventa:
\[
  \left( \Box - \me^{2 \left(d-1\right) \alpha \Phi}\frac{n^2}{L^2} \right) \phi_n = 0
\]
Confrontando con  l'equazione di Klein-Gordon per un campo massivo:
\[
  \left( \Box - m^2 \right) \phi = 0
\]
si ottiene che la massa dell'$ n $-esimo modo è $ m =  \me^{ \left(d-1\right) \alpha \Phi}\frac{n}{L} $.

\section*{Esercizio 4}

\subsection*{Testo}
A lezione abbiamo visto che usando l'ansatz di Kaluza-Klein \eqref{eq:metrica_problema_3} e l'indipendenza della metrica dalla coordinata z,
l'azione di Einstein-Hilbert in 5 dimensioni si riduce a:
\begin{equation}
  \label{eq:azione_EH}
  S = \frac{1}{16 \pi} \int \dddd x \left( R - \frac{1}{2} (\pd \Phi)^2 - \frac{1}{4} \me^{-\sqrt{3}\Phi} F^2 \right)
\end{equation}
Nonostante la complicatezza delle equazioni del moto che seguono dalla \eqref{eq:azione_EH} esiste una semplice tecnica per generare
delle soluzioni. Partiamo con lo spazio prodotto tra la metrica di Schwarzschild in 4 dimensioni e una linea con la metrica $ \d z^2 $,
\begin{equation}
  \label{eq:metrica_per_linea}
  \d {\hat{s}}^2_5 = - \left( 1 - 2mr^{-1} \right) \d t^2 + \left( 1 - 2mr^{-1} \right)^{-1} \d r^2 + r^2 \d \Omega^2 + \d z^2
\end{equation}
che è ovviamente una soluzione delle equazioni del moto in 5 dimensioni (perché?). Facciamo poi un boost lungo $ z $ con parametro $ \gamma $
(scrivere la trasformazione delle coordinate $ t, z $!), compattifichiamo $ z $ su un cerchio e paragoniamo con la \eqref{eq:metrica_problema_3}.
Si dimostri che in questo modo si ottiene per la metrica in 4 dimensioni
\begin{equation}
  \label{eq:metrica_in_4_dimensioni}
  \d{s^2_4} = \Delta^{-\frac{1}{2}}\left( 1 - 2mr^{-1} \right)\d t^2 + \Delta^{\frac{1}{2}} \left[ \left( 1 - 2mr^{-1} \right)^{-1}\d r^2 + r^2 \d \Omega^2 \right]
\end{equation}
con $ \Delta = 1 + 2mr^{-1}\sinh^2 \gamma $. Si faccia vedere inoltre che il dilatone e il campo di gauge sono dati da
\begin{equation}
  \label{eq:dilatone_gauge}
  \me^{\frac{2\Phi}{\sqrt{3}}} = \Delta \qquad A = \frac{m \sinh 2 \gamma}{r \Delta} dt
\end{equation}
Calcolare la carica elettrica del buco nero definita da:
\begin{equation}
  \label{eq:carica_buco_nero}
  Q = \frac{1}{4\pi} \int \me^{-\sqrt{3}\Phi} \hF
\end{equation}
dove $ {}^\star $ è il duale di Hodge di $ F $, e si integra su una 2-sfera per $ r \rightarrow + \infty $.
Dimostrare che
\begin{equation}
  \label{eq:carica_massa}
  Q = m \sinh 2 \gamma
\end{equation}
La massa $ M $ del buco nero può essere ottenuta dal termine leading dello sviluppo asintotico di $ g_{tt} $. Far vedere che questo dà:
\begin{equation}
  \label{eq:massa_buco_nero}
  M = \frac{m}{2} \left( 2 + \sinh^2 \gamma \right)
\end{equation}
Facoltativo: Calcolare la temperatura $ T $ del buco nero richiedendo l'assenza di singolarità coniche nella sezione euclidea, e dimostrare
che vale la prima legge
\begin{equation}
  \label{eq:prima_legge}
  \d M = T \d S + \phi \d Q
\end{equation}
con $ S = \mathcal{A}_{\mathrm{orizz.}} \slash 4 $ l'entropia di Bekenstein-Hawking ($ A_{\mathrm{orizz.}} $: area dell'orizzonte) e $ \phi = A_t \lvert_{\mathrm{orizz.}} $
il potenziale elettrico. \\
Altro compito facoltativo: Motivare la \eqref{eq:carica_buco_nero}.

\subsection*{Soluzione}
In quello che segue si utilizza la convenzione sui cappucci e sugli indici per spazi $ d $ e $ d+1 $-dimensionali: i cappucci e le lettere latine maiuscole
si riferiscono a grandezze nello spazio $ d+1 $-dimensionale, mentre le lettere greche e l'assenza di cappucci a quelle nello spazio $ d $-dimensionale.
La metrica \eqref{eq:metrica_per_linea} risolve le equazioni di Einstein nel vuoto perché il tensore di Ricci è a blocchi, infatti:
\[
  \hg_{M N}(x,z) =
  \begin{pmatrix}
    g_{\mu\nu}(x) & 0 \\
    0             & 1 \\
  \end{pmatrix}
\]
Siccome il tensore metrico è diagonale a blocchi è immediato calcolare l'inverso:
\[
  \hg^{M N} =
  \begin{pmatrix}
    g^{\mu\nu}(x) & 0 \\
    0             & 1 \\
  \end{pmatrix}
\]
Calcolando i simboli di Cristhoffel:
\[
  \hat{\Gamma}^{M}_{\;NO} = \frac{1}{2} \hg^{MP}\left( \partial_O \hg_{NZ} +  \partial_N \hg_{PO} -  \partial_P \hg_{ON}  \right)
\]
Qualora uno tra $ M, N $ e $ O $ sia $ z $ allora il simbolo è nullo in quanto nessun componente del tensore metrico
dipende da $ z $ e tutti i componenti di $ \hat{g} $ che contengono l'indice $ z $ sono costanti, quindi quando derivati si annullano.
A questo punto calcolando il tensore di curvatura:
\[
  \hat{R}^{M}_{\;NOP} = \partial_O \hat{\Gamma}^{M}_{\;PN} + \partial_P \hat{\Gamma}^{M}_{\;ON} + \hat{\Gamma}^{R}_{\;PN} \hat{\Gamma}^{M}_{\;OR}
  + \hat{\Gamma}^{R}_{\;OP} \hat{\Gamma}^{M}_{\;PR}
\]
Siccome i simboli di Cristhoffel sono nulli quando un indice è $ z $ anche le compoeneti del tensore di Riemann sono nulle quando uno
dei suoi indici è $ z $. Dato che il tensore di Ricci è $ R_{MN} = R^P_{\;MPN} $ quando uno dei suoi indici è $ z $ allora quella componente
è nulla, ciò significa che il Ricci è a blocchi con il blocco corrispondente a $ z $ nullo:
\[
  \hat{R}_{M N} =
  \begin{pmatrix}
    R_{\mu\nu} & 0 \\
    0             & 0 \\
  \end{pmatrix}
\]
Siccome $ R_{\mu\nu} = 0 $ perché la metrica in quattro dimensioni è quella di Schwarzschild vale che $ \hat{R}_{MN} = 0 $,
cioè sono soddisfatte le equazioni di Einstein in cinque dimensioni nel vuoto. \\ \\
Eseguo un boost di rapidità $ \gamma $:
\[
  \begin{pmatrix}
    t' \\
    z' \\
  \end{pmatrix}
  =
  \begin{pmatrix}
    \cosh \gamma & -\sinh \gamma \\
    -\sinh \gamma & \cosh \gamma \\
  \end{pmatrix}
  \begin{pmatrix}
    t \\
    z \\
  \end{pmatrix}
\]
L'inverso di questa trasformazione è, utilizzando le proprietà gruppali delle trasformazioni di Lorentz:
\[
  \begin{pmatrix}
    t \\
    z \\
  \end{pmatrix}
  =
  \begin{pmatrix}
    \cosh \gamma & \sinh \gamma \\
    \sinh \gamma & \cosh \gamma \\
  \end{pmatrix}
  \begin{pmatrix}
    t' \\
    z' \\
  \end{pmatrix}
\]
Da cui:
\[
  \begin{cases}
    \d t = \cosh \gamma \; \d t' + \sinh \gamma \; \d z' \\
    \d z = \sinh \gamma \; \d t' + \cosh \gamma \; \d z'
  \end{cases}
\]
E quindi:
\[
  \begin{cases}
    \d t^2 = \cosh^2 \gamma \; \d t'^2 + \sinh^2 \gamma \; \d z'^2 + 2 \sinh \gamma \cosh \gamma \; \d t' \d z' \\
    \d z^2 = \sinh^2 \gamma \; \d t'^2 + \cosh^2 \gamma \; \d z'^2 + 2 \sinh \gamma \cosh \gamma \; \d t' \d z'
  \end{cases}
\]
Questa trasformazione è un diffeomorfismo, quindi la metrica che si ottiene è ancora soluzione delle equazioni di Einstein in cinque
dimensioni.
Riscrivendo \eqref{eq:metrica_per_linea} con questo cambio di coordinate e sopprimendo per comodità i primati:
\begin{align*}
  \d{\hat{s}^2_5} & = - \left( 1 - \frac{2m}{r} \right) \d t^2 + \left( 1 - \frac{2m}{r} \right)^{-1} \d r^2 + r^2 \d \Omega^2 + \d z^2 = [\dots] = \\
                  & = - \left( 1 - \frac{2m}{r} \cosh^2 \gamma \right) \d t^2 + \frac{\d r^2}{1 - \frac{2m}{r}} + r^2 \d \Omega^2 +
                 \frac{2m}{r}\sinh 2\gamma \; \d t \d z + \left(1 + \frac{2m}{r}\sinh^2 \gamma \right) \d z^2
\end{align*}
Nel quale chiaramente si è posto con abuso di notazione $ \sinh 2 \gamma = \sinh (2\gamma) $ e si è utilizzata la relazione trigonometrica
iperbolica $ \sinh 2\gamma = 2 \sinh \gamma \cosh \gamma $. Si nota che per $ \gamma = 0 $,
cioè per un boost a velocità nulla (ovvero una trasformazione di identità) la metrica torna ad essere \eqref{eq:metrica_per_linea}. \\ \\
La metrica in cinque dimensioni è già indipendente da $ z $, quindi la riduzione consiste nel porre, in accordo con l'ansatz di Kaluza-Kelin:
\[
  \begin{cases}
    \hg_{\mu\nu} = \me^{2\alpha\Phi}g_{\mu\nu} + \me^{2\beta\Phi}A_\mu A_\nu \\
    \hg_{\mu z} = \me^{2\beta\Phi} A_\mu \\
    \hg_{zz} = \me^{2\beta\Phi}
  \end{cases}
\]
Cioè porre $ \d{\hat{s}^2_5} = \me^{2\alpha\Phi} \d{s^2_4} + \me^{2\beta\Phi}\left( \d z + A \right)^2 $. Sviluppando la teoria generale della riduzione
dimensionale di Kaluza-Kelin si scopre che una scelta furba per i parametri $ \alpha $ e $ \beta $ è:
\[
  \alpha = \frac{1}{\sqrt{2(d-1)(d-2)}} = \frac{1}{2\sqrt{3}} \qquad \beta = -(d-2)\alpha = - \frac{1}{\sqrt{3}}
\]
Tale scelta consente di ottenere l'azione di Einstein-Hilbert e del campo scalare con le normalizzazioni consuete. \\ \\ x
A questo punto si identifica il prefattore di $ \d z^2 $ con $ \me^{-\frac{2}{\sqrt{3}}\Phi} $:
\[
  \me^{-\frac{2}{\sqrt{3}}\Phi} = 1 + \frac{2m}{r}\sinh^2 \gamma
\]
Si definisce tale grandezza $ \Delta $, come richiede il testo dell'esercizio:
\[
  \Delta := 1 + \frac{2m}{r}\sinh^2 \gamma = \me^{-\frac{2}{\sqrt{3}}\Phi}
\]
Si nota che $ \Delta \to 1 $ se $ \gamma \to 0 $.
Con questa identificazione è possibile trovare anche il campo di gauge ricostruendo il termine $ \Delta (A_\mu \d x^\mu + \d z)^2 $. Siccome l'unico
doppio prodotto che compare nella metrica è $ \d t \d z $, significa che $ A $ ha solo la componente temporale. Aggiungendo e sottraendo dalla
metrica il termine $ \Delta \left( \frac{m \sinh 2\gamma}{r\Delta}\right)^2 \d t^2 $ si ricostruisce il quadrato desiderato:
\[
  \Delta \left[ \left( \frac{m\sinh2\gamma}{r\Delta} \right)^2 \d t^2 + \frac{2m \sinh 2\gamma}{r\Delta} \d t \d z + dz^2  \right]
\]
Identificando $ A = \frac{m \sinh 2 \gamma}{r \Delta} \d t $ si ottiene esattamente $ \me^{-\frac{2}{\sqrt{3}}\Phi} \left( A + \d z \right)^2 $. \\
Nella metrica rimangono i termini:
\[
  - \left( 1 - \frac{2m}{r} \cosh^2 \gamma \right) \d t^2 + \frac{\d r^2}{1 - \frac{2m}{r}} + r^2 \d \Omega^2 - \frac{4m^2 \sinh^2 \gamma \cosh^2 \gamma}{r^2 \Delta} \d t^2
\]
Si nota che $ \me^{2\alpha \Phi} = \me^{\frac{1}{\sqrt{3}}\Phi} = \Delta^{-\frac{1}{2}} $, per questo raccogliendo $ \Delta^{-\frac{1}{2}} $ si produce il corretto
prefattore davanti a $ \d s^2_4 $:
\[
  \Delta^{-\frac{1}{2}} \left[
    - \Delta^{\frac{1}{2}}\left( 1 - \frac{2m}{r} \cosh^2 \gamma \right) \d t^2 + \Delta^{\frac{1}{2}} \left( \frac{\d r^2}{1 - \frac{2m}{r}} + r^2 \d \Omega^2 \right)
    - \frac{4m^2 \sinh^2 \gamma \cosh^2 \gamma}{r^2 \Delta^{\frac{1}{2}}} \d t^2
  \right]
\]
Manipolo i termini con i $ \d t^2 $ moltiplicando a numeratore e a denominatore del primo termine $ \Delta^{\frac{1}{2}} $:
\begin{align*}
  &  - \frac{\Delta}{\Delta^{\frac{1}{2}}}\left( 1 - \frac{2m}{r} \cosh^2 \gamma \right) - \frac{4m^2 \sinh^2 \gamma \cosh^2 \gamma}{r^2 \Delta^{\frac{1}{2}}} = \\
  ={} & -\Delta^{-\frac{1}{2}} \left[ \left( 1 + \frac{2m}{r}\sinh^2 \gamma \right) \left( 1 - \frac{2m}{r}\cosh^2 \gamma \right)
        + \frac{4m^2 \sinh^2 \gamma \cosh^2 \gamma}{r^2} \right] \\
  = {} & -\Delta^{-\frac{1}{2}} \left( 1 - \frac{2m}{r} \right)
\end{align*}
Per cui complessivamente:
\[
  \d s^2_4 = -\frac{1}{\Delta^{\frac{1}{2}}}\left( 1 - \frac{2m}{r} \right)\d t^2 + \Delta^{\frac{1}{2}} \left( \frac{\d r^2}{1 - \frac{2m}{r}} + r^2 \d \Omega^2 \right)
\]
Introducendo la lapse-function $ N^2 = \Delta^{-\frac{1}{2}}\left( 1 - \frac{2m}{r} \right) $ la metrica assume la forma:
\begin{equation}
  \label{eq:metrica_con_lapse}
  \d s^2_4 = -N^2\d t^2 +  \frac{\d r^2}{N^2} + \Delta^{\frac{1}{2}} r^2 \d \Omega^2
\end{equation}
\\
Cerco ora la carica del buco nero con la formula del testo dell'esercizio \eqref{eq:carica_buco_nero}:
\[
  Q = \frac{1}{4\pi}\int \me^{-\sqrt{3}\Phi}\hF
\]
Definendo $ \tilde{F} = \me^{-\sqrt{3}\Phi} F = \Delta^{\frac{3}{2}} F $, dopo aver dimostrato che $ \htF = \me^{-\sqrt{3}\Phi} \hF $ la carica del buco nero è:
\[
  Q = \frac{1}{4\pi} \int \htF
\]
Il campo di gauge è:
\[
  A = \frac{\frac{m}{r} \sinh 2\gamma}{1 + \frac{2m}{r} \sinh^2 \gamma} \d{t}
\]
Mentre il tensore di Faraday:
\[
  F = \d A = -\frac{m \sinh 2\gamma}{(r + 2m \sinh^2 \gamma)^2} \d r \wedge \d{t}
\]
Mostro ora che $ \htF = \me^{-\sqrt{3}\Phi} \hF $.

\begin{lemma}
  Vale che  $ \htF_{\mu\nu} = \me^{-\sqrt{3}\Phi} \hF_{\mu\nu} = \Delta^{\frac{3}{2}}\frac{\sg}{2} \varepsilon_{\mu\nu\rho\sigma} F^{\rho\sigma}$.
\end{lemma}

\begin{proof}
  Sia $ \omega = \frac{1}{p!} \omega_{\mu_1\mu_2\dots\mu_p} \d x^{\mu_1} \wedge \d x^{\mu_2} \wedge \dots \d x^{\mu_p} \in \Lambda_p $ una $ p $-forma,
  allora il suo duale di Hodge in $ \Lambda_{q-p} $ è definito da:
  \[
    {}^\star \omega =  \frac{\sqrt{\sigma}}{p!(q-p)!} \omega_{\mu_1\mu_2\dots\mu_p} \; \varepsilon^{\mu_1\mu_2\dots\mu_p}_{\;\;\;\;\;\;\;\;\;\;\;\;\;\;\;\nu_{p+1}\dots\nu_{q}}
    \d x^{\nu_1} \wedge \d x^{\nu_2} \wedge \dots \d x^{\nu_q}
  \]
  Dove $ \sigma $ è il determinante della metrica indotta.
  Nel caso considerato $ p = 2 $ e $ q = 4 $ e $ \omega = F $:
  \[
    \hF = \frac{\sqrt{\sigma}}{4} F^{\mu\nu}\varepsilon_{\mu\nu\rho\sigma} \d x^\rho \wedge \d x^\sigma \quad \text{e} \quad
    \htF = \frac{\sqrt{\sigma}}{4} \tilde{F}^{\mu\nu}\varepsilon_{\mu\nu\rho\sigma} \d x^\rho \wedge \d x^\sigma
  \]
  Ma $ \tilde{F}^{\mu\nu} = \Delta^{\frac{3}{2}} F^{\mu\nu} $, quindi:
  \[
    \htF = \frac{\sqrt{\sigma}}{4} \Delta^{\frac{3}{2}}F^{\mu\nu}\varepsilon_{\mu\nu\rho\sigma} \d x^\rho \wedge \d x^\sigma = \Delta^{\frac{3}{2}} \hF
    = \Delta^{\frac{3}{2}} \frac{\sqrt{\sigma}}{4} F^{\mu\nu}\varepsilon_{\mu\nu\rho\sigma} \d x^\rho \wedge \d x^\sigma
  \]
  Sostanzialmente ho mostrato che l'operazione di dualità si comporta bene rispetto alla moltiplicazione di una 0-forma.
\end{proof}
\noindent
Calcolo $ \htF $ su una 2-sfera a $ t $ e $ r $ costanti. La metrica indotta è:
\[
  \d \sigma^2 = \Delta^{\frac{1}{2}}r^2(r) \d \Omega^2
\]
Inoltre noto che $ F^{rt} = - F_{rt} $ infatti:
\[
  F^{rt} = g^{r\alpha}g^{t\beta}F_{\alpha\beta} = g^{rr}g^{tt}F_{rt} = -N^{-2}N^2 F_{rt} = - F_{rt}
\]
Utilizzando la formula appena ricavata:
\begin{align*}
  \htF & = \Delta^{\frac{3}{2}}(r) \frac{\sqrt{\sigma}}{4} F^{\mu\nu}\;\varepsilon_{\mu\nu\rho\sigma} \d x^\rho \wedge \d x^\sigma
        = -\Delta^{\frac{3}{2}}(r) \sqrt{\sigma} F^{rt}  \d \theta \d \phi = \\
       & \overset{(\star)\footnotemark[1]}{=}
         \Delta^{\frac{3}{2}}(r) \sqrt{\sigma} F_{rt} \d \theta \d \phi
        = \Delta^{2}(r) {r^2} \sin\theta  \frac{m \sinh 2\gamma}{r^2 \Delta^2(r)} \d \theta \d \phi
        \overset{r\to +\infty}{\to} {m \sin\theta \sinh 2\gamma} \d \theta \d \phi
\end{align*}%
\footnotetext[1]{Il simbolo $\star$ è stato inserito per riferirsi successivamente a questo passaggio.}%
In cui nel primo passaggio il fattore $ 4 $ viene eliminato dall'antisimmetria di $ F^{\mu\nu} $ e di $ \d x^\rho \wedge \d x^\sigma $.
Per cui:
\[
  Q = \frac{1}{4\pi} \int_0^{2\pi} \int_0^\pi m \sin\theta \sinh 2\gamma \d \theta \d \phi = m \sinh 2\gamma
\]
Una giustificazione di perché la carica ha questa forma si ottiene osservando le equazioni del moto di questa azione:
\[
  \begin{cases}
    G_{\mu\nu} = \frac{1}{2} \left( \nabla_\mu \Phi \nabla_\nu \Phi - \frac{1}{2} \nabla_\alpha \Phi \nabla^\alpha \Phi
      g_{\mu\nu}\right) + \frac{1}{2}\me^{-\sqrt{3}\Phi} \left( F_{\mu\rho} F_\nu^{\;\rho} - \frac{1}{4}F^2g_{\mu\nu} \right) \\
    \nabla_\mu \left( \me^{-\sqrt{3}\Phi} F^{\mu\nu} \right) = 0 \\
    \Box \Phi = - \frac{\sqrt{3}}{4}\me^{-\sqrt{3}\Phi} F^2
  \end{cases}
\]
Questo è un sistema di Einstein-Maxwell-scalare. Considerando una teoria puramente elettromagnetica, cioè descritta dall'azione
\[
  S = \int \dddd \sg \frac{1}{4}F_{\mu\nu}F^{\mu\nu}
\]
Questa azione conduce alle equazioni del moto $ \nabla_\mu F^{\mu\nu} = 0 $. Per aggiungere le correnti si accoppia l'azione con un termine
del tipo $ 4 \pi A_\mu J^\mu $ e ciò porta alle equazioni del moto: $ \nabla_\mu F^{\mu\nu} = 4 \pi J^\nu $; on shell esiste quindi una relazione
tra le correnti e il tensore elettromagnetico. La carica contenuta in volume $ V $ di bordo $ \pd V $ con elemento di volume orientato $ \d S_\mu $ è:
\[
  Q = \int_V J^\mu \d S_\mu
\]
Utilizzando l'equazione del moto:
\[
  Q = \frac{1}{4\pi}\int_V \nabla_\nu F^{\nu\mu} \d S_\mu
\]
Con il teorema della divergenza, indicando con $ \d S_{\mu\nu} $ l'elemento di superficie orientata, ovvero l'antisimmetrizzazione tra i versori
normali a $ \pd V $:
\[
  Q = \frac{1}{8\pi} \oint_{\pd V} F^{\nu\mu} \d S_{\mu\nu} =  - \frac{1}{8\pi} \oint_{\pd V} F_{\mu\nu} \d S^{\mu\nu}
\]
Assumendo che anche in una teoria Maxwell-scalare valgano le stesse conclusioni con $ F \mapsto \tilde{F} $, vale che:
\[
  Q = - \frac{1}{8\pi} \oint_{\pd V} \tilde{F}_{\mu\nu} \d S^{\mu\nu}
\]
Se mostro che $ \tilde{F}_{\mu\nu} \d S^{\mu\nu} = - 2 \htF $ ottengo che la carica si calcola con la relazione \eqref{eq:carica_buco_nero}.
Ma:
\[
  \d S^{\mu\nu} = \left( v^\mu u^\nu - u^\mu v^\nu \right) \sqrt{\sigma} \d \theta \d \phi \quad \text{con} \quad u = N^{-1}\pd_t \; v = N \pd_r
\]
L'unica componente di interesse è:
\[
  \d S^{rt} = \sqrt{\sigma} \d \theta \d \phi = - \d S^{tr}
\]
Ma:
\begin{align*}
  \tilde{F}_{\mu\nu} \d S^{\mu\nu} = \tilde{F}_{rt} \d S^{rt} + \tilde{F}_{tr} \d S^{tr} = 2 \tilde{F}_{rt} \d S^{rt} = 2 \Delta^{\frac{3}{2}} F_{rt} \sqrt{\sigma} \d \theta \d \phi
\end{align*}
Questo è esattamente $ 2(\star) $, dove $ (\star) $ indica l'espressione contrassegnata nel calcolo di $ \htF $. Si è dimostrato quello che si voleva
dimostrare a meno di un segno di differenza. In effetti questo ragionamento permette di identificare la carica a meno di una costante moltiplicativa,
in quanto le equazioni del moto della teoria Maxwell-scalare sono invarianti per trasformazioni $ F \mapsto \lambda F $, e questo porta all'ambiguità,
la quale, in fondo, deriva dal fatto che se $ Q $ è una carica conservata sulle soluzioni dell'equazione del moto anche $ \lambda Q $ lo è.
Questo potrebbe spiegare i mismatch tra i segni. Una seconda possibilità non trascurabile per spiegare questo mismatch sono errori di calcolo. \\ \\
Per calcolare la massa al posto di calcolare un integrale di Komar posso imporre che la metrica in gravità debole si riconduca al giusto limite Newtoniano.
Per far ciò osservo il moto di una particella di test nella regione asintotica. L'equazione delle geodetiche è:
\[
  \frac{\dd{x^\mu}}{\d{\tau^2}} + \Gamma^\mu_{\; \rho\sigma} \frac{\d{x^\rho}}{\d{\tau}} \frac{\d{x^\sigma}}{\d{\tau}} = 0
\]
In approssimazione classica, ovvero nel limite di velocità molto più piccole di quelle della luce,  vale che $ (\dot{x^\rho}) \approx (1,0,0,0) $ e
che $ \tau \approx t $, quindi l'equazione delle geodetiche diventa:
\[
  \frac{\dd{x^\mu}}{\d{t^2}} = -\Gamma^\mu_{\; tt} = -\frac{1}{2}g^{\mu\alpha} (\pd_t g_{\alpha t} + \pd_t g_{t \alpha} - \pd_\alpha g_{tt} )
\]
Siccome la metrica non dipende esplicitamente dal tempo le derivate temporali sono nulle:
\[
  \frac{\dd{x^\mu}}{\d{t^2}} = \frac{1}{2}g^{\mu i} \pd_i g_{tt} \quad \rightarrow \quad  \frac{\dd{x^i}}{\d{t^2}} = \frac{1}{2}g^{i j}\pd_j g_{tt}
\]
Fermando lo sviluppo ai termini di ordine $ \mathcal{O}(\frac{1}{r}) $ si trova $ g_{tt} = - 1 + \frac{m}{r}(2 + \sinh^2 \gamma) $, da cui:
\[
  \frac{\dd{x^i}}{\d{t^2}} = \frac{1}{2} \pd_i \left(\frac{m}{r}(2 + \sinh^2 \gamma) \right)
\]
In meccanica classica il moto di una massa in un campo gravitazionale generato da una massa puntiforme è descritto da:
\[
  \frac{\dd{x^i}}{\d{t^2}} = \pd_i \frac{M}{r}
\]
Per cui l'identificazione $  M = \frac{m}{2} \left( 2 + \sinh^2 \gamma \right) $. \\ \\
%
Utilizzando la metrica nella forma \eqref{eq:metrica_con_lapse} calcolo la temperatura passando alla sezione euclidea, cioè mandando $ t \to -i \tau $:
\[
  \d s^2_E = N^2 \d \tau^2 + \frac{\d r^2}{N^2} + \Delta^{\frac{1}{2}}r^2 \d \Omega^2
\]
L'orizzonte è definito da $ N(r_H) = 0 $, quindi si ottiene $ r_H = 2m $, per il quale si nota che $ \Delta(r_H) = 1 + \frac{2m}{r_H}\sinh^2 \gamma = \cosh^2 \gamma $.
Sviluppando $ N^2 $ vicino all'orizzonte $ r_H $:
\[
  N^2 = \cancel{N^2(r_H)} + \frac{\d N^2}{\d r} \Biggr\lvert_{r_H} \left( r - r_H \right) + \dots \approx \alpha \left( r - r_H \right)
\]
Nella quale si è definita la costante $ \alpha =  \frac{\d N^2}{\d r} \lvert_{r_H} $.
Definendo anche $ \rho^2 = r - r_H $ si ha:
\[
  \d r = 2 \rho \d \rho \quad \rightarrow \quad \d r^2 = 4 \rho^2 \d \rho^2
\]
Sostituendo lo sviluppo nella metrica:
\[
  \d s^2_E = \alpha \d \tau^2 + \frac{4 \rho^2 \d \rho^2}{\alpha} + \Delta^{\frac{1}{2}}(r_H)r_H^2 \d \Omega^2
\]
Considerando solo i termini singolari:
\[
  \d s^2_E \to \frac{4}{\alpha} \left[ \d \rho^2 + \rho^2 \d{ \left( \frac{\alpha \tau}{2} \right)^2} \right]
\]
La metrica non presenta singolarità coniche se:
\[
  \tau \sim \tau + \frac{4\pi}{\alpha}
\]
Quindi:
\[
  \beta = \frac{1}{T} = \frac{4\pi}{\alpha}
\]
Ma:
\[
  \alpha =  \frac{\d N^2}{\d r} \Biggr\lvert_{r_H} = \frac{m \left( 2r + (2m + r) \sinh^2 \gamma \right)}{r^2 (r + 2m \sinh^2\gamma) \sqrt{1 + \frac{2m\sinh^2 \gamma}{r}}}
  \Biggr\lvert_{r_H} = \frac{1}{2m \cosh \gamma}
\]
Quindi la temperatura vale:
\[
  T = \frac{1}{8 \pi m \cosh \gamma}
\]
L'area dell'orizzonte $ r_H $ è:
\[
  \mathcal{A}_{\mathrm{orizz.}} = \int \sqrt{g_{\theta\theta}g_{\phi\phi}}\; r^2 \sin\theta \d{\theta} \d{\phi} = 4 r_H^2 \pi \Delta^{\frac{1}{2}}(r_H) = 16 m^2 \pi \cosh \gamma
\]
Da cui si ottiene l'entropia di Bekenstein-Hawking:
\[
  S = \frac{\mathcal{A}}{4} =  4 m^2 \pi \cosh \gamma
\]
Il potenziale elettrico invece è:
\[
  \phi = A_t \lvert_{\mathrm{orizz.}} = \frac{m \sinh 2 \gamma}{r_H \Delta(r_H)} = \frac{\cosh \gamma \sinh \gamma}{\cosh^2 \gamma} = \tanh \gamma
\]
Ricapitolando le proprietà elettro-termomeccaniche del buco nero:
\begin{align*}
  M & = m \left( 1 + \frac{1}{2}\sinh^2 \gamma \right) \\
  T & = \frac{1}{8 \pi m \cosh \gamma} \\
  S & = 4 \pi m^2 \cosh \gamma \\
  \phi & = \tanh \gamma \\
  Q & = 2m \cosh \gamma \sinh \gamma
\end{align*}
Da cui si può verificare la legge di Smarr $ M = 2 TS + \frac{1}{4}\phi Q $. Questa legge ha una forma inattesa, in quanto
ci si aspettava $ M = 2 TS + \phi Q $, che riflette il fatto che l'entropia ha dimensioni $ [M]^2 $ mentre la carica $ [M] $.
Se si ridefinisce o la carica o il potenziale con un fattore $ 4 $ correttivo si verifica esplicitamente la prima
legge delle meccanica dei buchi neri: %
\begin{gather*}
  \d M  = \d m \left( 1 + \frac{1}{2}\sinh^2\gamma \right) \\
  T \d S + \phi \d  Q  = \frac{1}{8 \pi m \cosh \gamma} 8 \pi m \cosh \gamma \d m +  \frac{1}{4} \frac{\sinh}{\cosh} \cosh \gamma \sinh \gamma \d m  = \d m \left( 1 + \frac{1}{2}\sinh^2\gamma \right)
\end{gather*}
Da cui:
\[
  \d M = T \d S + \phi \d Q
\]
% Vale cioè che:
% \[
%   M(\lambda^2 S, \lambda Q) = \lambda M \qquad \forall \lambda \in \mathbb{R}
% \]
% Derivando per $ \lambda $ e poi ponendo $ \lambda = 1 $:
% \[
%   M = 2 \frac{\pd M}{\pd S} \Biggr \lvert_Q S + \frac{\pd M}{\pd Q} \Biggr \lvert_S Q
% \]
% Ma vale la legge di Smarr, differenziandola:
% \[
%   \d M = 2 T \d S + \phi \d Q
% \]
% Da cui si trova:
% \[
%   \frac{\pd M}{\pd S} \Biggr \lvert_Q = T \qquad \frac{\pd M}{\pd Q} \Biggr \lvert_S = \phi
% \]

% Per definizione di differenziale:
% \[
%   M = \frac{\pd M}{\pd S} \Biggr \lvert_Q S + \frac{\pd M}{\pd Q} \Biggr \lvert_S Q = T \d S + \phi \d Q
% \]
% Esplicitamente:
\end{document}

%%% Local Variables:
%%% mode: latex
%%% TeX-master: t
%%% End:
